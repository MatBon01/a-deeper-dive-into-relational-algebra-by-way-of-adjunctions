\begin{comment}
The abstract is a very brief summary of the report's contents. It should be no more than half a page long. Somebody unfamiliar with your project should have a good idea of what it's about having read the abstract alone and will know whether it will be of interest to them. Note that the abstract is a summary of the entire project including its conclusions. A common mistake is to provide only introductory elements in the abstract without saying what has been achieved.
\end{comment}


\begin{abstract}
In order to evaluate alternative equijoin implementation in Haskell, this paper
introduced the \relation{JOINBENCH} relation and surrounding methodology and
tooling. The \relation{JOINBENCH} relation is a scheme designed for the
synthesis of data optimised to help evaluate the performance of equijoins in a
variety of scenarios. Furthermore, this paper presents a low-level library that
helps users customise and define their own synthetic data sources for future
benchmarking purposes.

In their distinguished paper
\relalg{}~\cite{RelationalAlgebraByWayOfAdjunctions} it was noted that the
monadic structure of bulk types can help explain most of relational algebra.
Using this structure, the authors designed a new method to facilitate
the use of monad comprehensions in an efficient implementation of equijoins of
relational databases. This project presents an implementation of such a system
and an evaluation of the performance such query optimisations carry using the
bespoke tooling described above. 
\end{abstract}
