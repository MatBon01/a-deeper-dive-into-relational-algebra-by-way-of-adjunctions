\chapter{Project Plan} % 1-2 pages
\begin{comment}
You should explain what needs to be done in order to complete the project and roughly what you expect the timetable to be. Don’t forget to include the project write-up (the final report), as this is a major part of the exercise. It’s important to identify key milestones and also fall-back positions, in case you run out of time.  You should also identify what extensions could be added if time permits.  The plan should be complete and should include those parts that you have already addressed (make it clear how far you have progressed at the time of writing).  This material will *not* appear in the final report.
\end{comment}

The project has a clear starting point and a wide range of potential branches to eventually explore. This allows the project to venture down many different specific roots, ranging in application focused to theoretically based.
We outline the stages the project can take below.
\section{Initial research and literature review}
\section{Implementation of the differing equijoin algorithms}
\subsection{Implementation}
\subsection{Evaluation and analysis}
\section{Theoretical analysis of the remaining relational algebra}
\section{Reporting the project}
% Report due Monday 19 June 2022
\section{Possible extensions}
\subsection{Progress in optimising other aspects of relational algebra}
\subsection{A full implementation}
\subsection{Applications of related fields to relational calculus}
