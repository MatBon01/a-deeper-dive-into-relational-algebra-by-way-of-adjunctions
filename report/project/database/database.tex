\chapter{Database implementation}\label{chap:database}
This chapter will describe the implementation of the database system outlined in
\relalg{}. The aims of the chapter are to highlight key implementations of the
algorithms of the paper. The purpose of implementing the structures in the paper
are to benchmark and so a high level pragmatic approach is taken in order to
create a working system that can easily be benchmarked.

The structure of the database management system implemented in this
project is a library based off of the code in
\relalg{}~\cite{RelationalAlgebraByWayOfAdjunctions} especially the appendix.
This chapter describes the code needed to create a fully functioning
system as well as highlights my contributions.

\paragraph{Bags} In order to construct the database system an
implementation of a bag (also known as multiset) data structure is
required. Fleshing out the $Bag$ type is one of the main challenges and
contributions of this project in the area of implementing the database. We
base the bag type off a list as the list allows multiple elements and
define an equality function.

\begin{hscode}\SaveRestoreHook
\column{B}{@{}>{\hspre}l<{\hspost}@{}}%
\column{5}{@{}>{\hspre}l<{\hspost}@{}}%
\column{7}{@{}>{\hspre}l<{\hspost}@{}}%
\column{9}{@{}>{\hspre}l<{\hspost}@{}}%
\column{11}{@{}>{\hspre}l<{\hspost}@{}}%
\column{13}{@{}>{\hspre}l<{\hspost}@{}}%
\column{17}{@{}>{\hspre}l<{\hspost}@{}}%
\column{26}{@{}>{\hspre}c<{\hspost}@{}}%
\column{26E}{@{}l@{}}%
\column{29}{@{}>{\hspre}l<{\hspost}@{}}%
\column{E}{@{}>{\hspre}l<{\hspost}@{}}%
\>[B]{}\mathbf{newtype}\;\Conid{Bag}\;\Varid{a}\mathrel{=}\Conid{Bag}\;\{\mskip1.5mu \Varid{elements}\mathbin{::}[\mskip1.5mu \Varid{a}\mskip1.5mu]\mskip1.5mu\}{}\<[E]%
\\[\blanklineskip]%
\>[B]{}\mathbf{instance}\;(\Conid{Eq}\;\Varid{a})\Rightarrow \Conid{Eq}\;(\Conid{Bag}\;\Varid{a})\;\mathbf{where}{}\<[E]%
\\
\>[B]{}\hsindent{5}{}\<[5]%
\>[5]{}\Varid{b1}\equiv \Varid{b2}\mathrel{=}\Varid{eq'}\;(\Varid{elements}\;\Varid{b1})\;(\Varid{elements}\;\Varid{b2}){}\<[E]%
\\
\>[5]{}\hsindent{2}{}\<[7]%
\>[7]{}\mathbf{where}{}\<[E]%
\\
\>[7]{}\hsindent{2}{}\<[9]%
\>[9]{}\Varid{eq'}\mathbin{::}{}\<[17]%
\>[17]{}(\Conid{Eq}\;\Varid{a})\Rightarrow [\mskip1.5mu \Varid{a}\mskip1.5mu]\to [\mskip1.5mu \Varid{a}\mskip1.5mu]\to \Conid{Bool}{}\<[E]%
\\
\>[7]{}\hsindent{2}{}\<[9]%
\>[9]{}\Varid{eq'}\;(\Varid{x}\mathbin{:}\Varid{xs})\;\Varid{ys}{}\<[26]%
\>[26]{}\mathrel{=}{}\<[26E]%
\>[29]{}\neg \;(\Varid{null}\;\Varid{ys2})\mathrel{\wedge}\Varid{eq'}\;\Varid{xs}\;(\Varid{ys1}\plus \Varid{tail}\;\Varid{ys2}){}\<[E]%
\\
\>[9]{}\hsindent{2}{}\<[11]%
\>[11]{}\mathbf{where}{}\<[E]%
\\
\>[11]{}\hsindent{2}{}\<[13]%
\>[13]{}(\Varid{ys1},\Varid{ys2})\mathrel{=}\Varid{break}\;(\equiv \Varid{x})\;\Varid{ys}{}\<[E]%
\\
\>[7]{}\hsindent{2}{}\<[9]%
\>[9]{}\Varid{eq'}\;[\mskip1.5mu \mskip1.5mu]\;[\mskip1.5mu \mskip1.5mu]{}\<[26]%
\>[26]{}\mathrel{=}{}\<[26E]%
\>[29]{}\Conid{True}{}\<[E]%
\\
\>[7]{}\hsindent{2}{}\<[9]%
\>[9]{}\Varid{eq'}\;\anonymous \;\anonymous {}\<[26]%
\>[26]{}\mathrel{=}{}\<[26E]%
\>[29]{}\Conid{False}{}\<[E]%
\ColumnHook
\end{hscode}\resethooks


\noindent
It is worth noting that the equality must be defined in such a way that does not
pay attention to the order of elements in the bag. We first create a helper
function $eq'$ that checks equality on lists $xs$ and $ys$ up to permutation.
Given that $xs$ is not empty we check whether $ys$ is non-empty as otherwise
equality would not be possible. If $ys$ has an element, we split the list into
two parts where the head of $xs$ is the first element of the second half of
$ys$. We can then `remove' that element from $ys$, reconstruct the list and
recursively check equality.
