\chapter{Database implementation}\label{chap:database}
This chapter will describe the implementation of the database system outlined in
\relalg{}. The aims of the chapter are to highlight key implementations of the
algorithms of the paper. The purpose of implementing the structures in the paper
are to benchmark and so a high level pragmatic approach is taken in order to
create a working system that can easily be benchmarked. 

The structure of the database management system implemented in this
project is a library based off of the code in
\relalg{}~\cite{RelationalAlgebraByWayOfAdjunctions} especially the appendix.
This chapter describes the code needed to create a fully functioning
system as well as highlights my contributions. It is worth noting that
a lot of effort went into the integration of the suggested algorithms and code
into a fully functional system that is difficult to present in a paper, this
section will focus on key parts of the system (despite often overlapping
\relalg{}) in order to describe what is being benchmarked in subsequent
chapters.

\paragraph{Bags} In order to construct the database system an
implementation of a $Bag$ (also known as multiset) data structure is
required. Fleshing out the $Bag$ type is one of the main challenges and
contributions of this project in the area of implementing the database. We
base the $Bag$ type off a list as the list allows multiple elements and
define an equality function.

\begin{hscode}\SaveRestoreHook
\column{B}{@{}>{\hspre}l<{\hspost}@{}}%
\column{5}{@{}>{\hspre}l<{\hspost}@{}}%
\column{7}{@{}>{\hspre}l<{\hspost}@{}}%
\column{9}{@{}>{\hspre}l<{\hspost}@{}}%
\column{11}{@{}>{\hspre}l<{\hspost}@{}}%
\column{13}{@{}>{\hspre}l<{\hspost}@{}}%
\column{17}{@{}>{\hspre}l<{\hspost}@{}}%
\column{26}{@{}>{\hspre}c<{\hspost}@{}}%
\column{26E}{@{}l@{}}%
\column{29}{@{}>{\hspre}l<{\hspost}@{}}%
\column{E}{@{}>{\hspre}l<{\hspost}@{}}%
\>[B]{}\mathbf{newtype}\;\Conid{Bag}\;\Varid{a}\mathrel{=}\Conid{Bag}\;\{\mskip1.5mu \Varid{elements}\mathbin{::}[\mskip1.5mu \Varid{a}\mskip1.5mu]\mskip1.5mu\}{}\<[E]%
\\[\blanklineskip]%
\>[B]{}\mathbf{instance}\;(\Conid{Eq}\;\Varid{a})\Rightarrow \Conid{Eq}\;(\Conid{Bag}\;\Varid{a})\;\mathbf{where}{}\<[E]%
\\
\>[B]{}\hsindent{5}{}\<[5]%
\>[5]{}\Varid{b1}\equiv \Varid{b2}\mathrel{=}\Varid{eq'}\;(\Varid{elements}\;\Varid{b1})\;(\Varid{elements}\;\Varid{b2}){}\<[E]%
\\
\>[5]{}\hsindent{2}{}\<[7]%
\>[7]{}\mathbf{where}{}\<[E]%
\\
\>[7]{}\hsindent{2}{}\<[9]%
\>[9]{}\Varid{eq'}\mathbin{::}{}\<[17]%
\>[17]{}(\Conid{Eq}\;\Varid{a})\Rightarrow [\mskip1.5mu \Varid{a}\mskip1.5mu]\to [\mskip1.5mu \Varid{a}\mskip1.5mu]\to \Conid{Bool}{}\<[E]%
\\
\>[7]{}\hsindent{2}{}\<[9]%
\>[9]{}\Varid{eq'}\;(\Varid{x}\mathbin{:}\Varid{xs})\;\Varid{ys}{}\<[26]%
\>[26]{}\mathrel{=}{}\<[26E]%
\>[29]{}\neg \;(\Varid{null}\;\Varid{ys2})\mathrel{\wedge}\Varid{eq'}\;\Varid{xs}\;(\Varid{ys1}\plus \Varid{tail}\;\Varid{ys2}){}\<[E]%
\\
\>[9]{}\hsindent{2}{}\<[11]%
\>[11]{}\mathbf{where}{}\<[E]%
\\
\>[11]{}\hsindent{2}{}\<[13]%
\>[13]{}(\Varid{ys1},\Varid{ys2})\mathrel{=}\Varid{break}\;(\equiv \Varid{x})\;\Varid{ys}{}\<[E]%
\\
\>[7]{}\hsindent{2}{}\<[9]%
\>[9]{}\Varid{eq'}\;[\mskip1.5mu \mskip1.5mu]\;[\mskip1.5mu \mskip1.5mu]{}\<[26]%
\>[26]{}\mathrel{=}{}\<[26E]%
\>[29]{}\Conid{True}{}\<[E]%
\\
\>[7]{}\hsindent{2}{}\<[9]%
\>[9]{}\Varid{eq'}\;\anonymous \;\anonymous {}\<[26]%
\>[26]{}\mathrel{=}{}\<[26E]%
\>[29]{}\Conid{False}{}\<[E]%
\ColumnHook
\end{hscode}\resethooks


\noindent
It is worth noting that the equality must be defined in such a way that does not
pay attention to the order of elements in the $Bag$. We first create a helper
function $eq'$ that checks equality on lists $xs$ and $ys$ up to permutation.
Given that $xs$ is not empty we check whether $ys$ is non-empty as otherwise
equality would not be possible. If $ys$ has an element, we split the list into
two parts where the head of $xs$ is the first element of the second half of
$ys$. We can then `remove' that element from $ys$, reconstruct the list and
recursively check equality.

As well as defining equality it is also important create instances of $Monad$,
$Functor$ and $CMonoid$ type classes for the $Bag$ type. A $Bag$'s status as a
$Functor$ allows $fmap$s
(the parallel to a projection as seen in \fref{sec:background:dbrep}) to be performed on
them and $Monad$ type is important to allow monad comprehensions which was a key
motivating factor for the paper \relalg{}. Details of the implementations for these instances are
left out of this paper as they simply follow the structure of using the instances on
the underlying list and translating the results back to a bag. A $CMonoid$ is a new type class
defined to distinguish commutative monoids from regular monoids; commutativity
is important as for a bulk type to allow for aggregation (which we expect from a
bag type) it must form a commutative monoid which it does under the union
operation. To allow for extended functionality and testing other interesting
types were given instances of $CMonoid$s such as $Sum\;k$ and $Product\;k$.
Moreover, $filter$
and $union$ functions for $Bag$s must be defined to cover the full extent of
relational algebra but their implementations are omitted as they are either
elementary or have been detailed in \relalg{}.

\paragraph{Pointed set} In order to create an indexed structure we must first
implement structures for a pointed set. The pointed set gives us a `null'
element to use for keys where there are no value present and therefore is vital
to a future indexed structure. The type class for a pointed set suggested in
\relalg{} is given below.

\begin{hscode}\SaveRestoreHook
\column{B}{@{}>{\hspre}l<{\hspost}@{}}%
\column{5}{@{}>{\hspre}l<{\hspost}@{}}%
\column{13}{@{}>{\hspre}c<{\hspost}@{}}%
\column{13E}{@{}l@{}}%
\column{17}{@{}>{\hspre}l<{\hspost}@{}}%
\column{E}{@{}>{\hspre}l<{\hspost}@{}}%
\>[B]{}\mathbf{class}\;\Conid{PointedSet}\;\Varid{a}\;\mathbf{where}{}\<[E]%
\\
\>[B]{}\hsindent{5}{}\<[5]%
\>[5]{}\Varid{null}{}\<[13]%
\>[13]{}\mathbin{::}{}\<[13E]%
\>[17]{}\Varid{a}{}\<[E]%
\\
\>[B]{}\hsindent{5}{}\<[5]%
\>[5]{}\Varid{isNull}{}\<[13]%
\>[13]{}\mathbin{::}{}\<[13E]%
\>[17]{}\Varid{a}\to \Conid{Bool}{}\<[E]%
\ColumnHook
\end{hscode}\resethooks


\noindent
It is important that our $Bag$ type is a pointed set so that we can index our
database by key, thereby assigning bags to each key value in the relation. An
original modified instance of the $PointedSet$ type class for $Bag$ is given
below. Additionally, an instance for $Maybe$ is given as it is an extremely
useful way to promote any type to a $PointedSet$ and was implemented to help
with testing.

\begin{hscode}\SaveRestoreHook
\column{B}{@{}>{\hspre}l<{\hspost}@{}}%
\column{5}{@{}>{\hspre}l<{\hspost}@{}}%
\column{13}{@{}>{\hspre}c<{\hspost}@{}}%
\column{13E}{@{}l@{}}%
\column{16}{@{}>{\hspre}l<{\hspost}@{}}%
\column{22}{@{}>{\hspre}c<{\hspost}@{}}%
\column{22E}{@{}l@{}}%
\column{25}{@{}>{\hspre}l<{\hspost}@{}}%
\column{E}{@{}>{\hspre}l<{\hspost}@{}}%
\>[B]{}\mathbf{instance}\;\Conid{PointedSet}\;(\Conid{Bag}\;\Varid{a})\;\mathbf{where}{}\<[E]%
\\
\>[B]{}\hsindent{5}{}\<[5]%
\>[5]{}\Varid{null}{}\<[22]%
\>[22]{}\mathrel{=}{}\<[22E]%
\>[25]{}\Varid{\Conid{Bag}.empty}{}\<[E]%
\\
\>[B]{}\hsindent{5}{}\<[5]%
\>[5]{}\Varid{isNull}\;(\Conid{Bag}\;[\mskip1.5mu \mskip1.5mu]){}\<[22]%
\>[22]{}\mathrel{=}{}\<[22E]%
\>[25]{}\Conid{True}{}\<[E]%
\\
\>[B]{}\hsindent{5}{}\<[5]%
\>[5]{}\Varid{isNull}\;\anonymous {}\<[22]%
\>[22]{}\mathrel{=}{}\<[22E]%
\>[25]{}\Conid{False}{}\<[E]%
\\[\blanklineskip]%
\>[B]{}\mathbf{instance}\;\Conid{PointedSet}\;(\Conid{Maybe}\;\Varid{a})\;\mathbf{where}{}\<[E]%
\\
\>[B]{}\hsindent{5}{}\<[5]%
\>[5]{}\Varid{null}{}\<[13]%
\>[13]{}\mathrel{=}{}\<[13E]%
\>[16]{}\Conid{Nothing}{}\<[E]%
\\
\>[B]{}\hsindent{5}{}\<[5]%
\>[5]{}\Varid{isNull}{}\<[13]%
\>[13]{}\mathrel{=}{}\<[13E]%
\>[16]{}\Varid{isNothing}{}\<[E]%
\ColumnHook
\end{hscode}\resethooks


\noindent
The advantage of pattern matching instead of comparing when defining a
$PointedSet$ for $Bag$ is that it does not force $Eq$ on the type in the $Bag$.
This is unlikely to be a problem in the very controlled environment of the
benchmark but I appreciated the more general approach so modified the instance
to allow for this behaviour. To restate the power of letting $Maybe;a$ be an
instance of $PointedSet$, it allows us to promote any type to a $PointedSet$ and
therefore be the value to a map. This means that when we want a `null' value for
any type we can accept $Nothing$ but in return any non-null value must be of the
form $Just a$ where $a$ has the promoted type.

\paragraph{Finite map} Finally we define a finite map. As the map is the most
complicated structure to implement and explain we will only present one
implementation also found in \relalg{}. A few others also from the paper were
defined in the project for testing however they are not presented here. Only a
handful of examples were implemented as much of the implementation was out of
the scope of an early implementation of this database system with the intention
of benchmarking. As will be discussed in \fref{sec:benchmark:joinbench} only
integer keys will be used for indexing and so it suffices to implement a map
whose key is $Word16$, representing a map with constant space (all of which can
be indexed by a 16 bit unsigned integer). The implementation of the map as
similar to that given by the paper is below.

\begin{hscode}\SaveRestoreHook
\column{B}{@{}>{\hspre}l<{\hspost}@{}}%
\column{5}{@{}>{\hspre}l<{\hspost}@{}}%
\column{7}{@{}>{\hspre}l<{\hspost}@{}}%
\column{9}{@{}>{\hspre}l<{\hspost}@{}}%
\column{27}{@{}>{\hspre}c<{\hspost}@{}}%
\column{27E}{@{}l@{}}%
\column{30}{@{}>{\hspre}l<{\hspost}@{}}%
\column{E}{@{}>{\hspre}l<{\hspost}@{}}%
\>[B]{}\mathbf{instance}\;\Conid{Key}\;\Conid{Word16}\;\mathbf{where}\;\mbox{\onelinecomment  constant type (array indexed by 16 bit word)}{}\<[E]%
\\
\>[B]{}\hsindent{5}{}\<[5]%
\>[5]{}\mathbf{newtype}\;\Conid{Map}\;\Conid{Word16}\;\Varid{v}{}\<[27]%
\>[27]{}\mathrel{=}{}\<[27E]%
\>[30]{}\Conid{A}\;(\Conid{Array}\;\Conid{Word16}\;\Varid{v})\;\mathbf{deriving}\;(\Conid{Eq},\Conid{Show}){}\<[E]%
\\
\>[B]{}\hsindent{5}{}\<[5]%
\>[5]{}\Varid{empty}{}\<[27]%
\>[27]{}\mathrel{=}{}\<[27E]%
\>[30]{}\Conid{A}\;(\Varid{accumArray}\;(\mathbin{\char92 \char95 }\Varid{x}\to \Varid{x})\;\Varid{\Conid{PointedSet}.null}\;(\mathrm{0},\mathrm{2}\mathbin{\uparrow}\mathrm{16}\mathbin{-}\mathrm{1})\;[\mskip1.5mu \mskip1.5mu]){}\<[E]%
\\
\>[B]{}\hsindent{5}{}\<[5]%
\>[5]{}\Varid{isEmpty}\;(\Conid{A}\;\Varid{a}){}\<[27]%
\>[27]{}\mathrel{=}{}\<[27E]%
\>[30]{}\Varid{all}\;\Varid{isNull}\;(\Varid{elems}\;\Varid{a}){}\<[E]%
\\
\>[B]{}\hsindent{5}{}\<[5]%
\>[5]{}\Varid{single}\;(\Varid{k},\Varid{v}){}\<[27]%
\>[27]{}\mathrel{=}{}\<[27E]%
\>[30]{}\Conid{A}\;(\Varid{accumArray}\;(\mathbin{\char92 \char95 }\Varid{x}\to \Varid{x})\;\Varid{\Conid{PointedSet}.null}\;(\mathrm{0},\mathrm{2}\mathbin{\uparrow}\mathrm{16}\mathbin{-}\mathrm{1})\;[\mskip1.5mu (\Varid{k},\Varid{v})\mskip1.5mu]){}\<[E]%
\\
\>[B]{}\hsindent{5}{}\<[5]%
\>[5]{}\Varid{merge}\;(\Conid{A}\;\Varid{a1},\Conid{A}\;\Varid{a2}){}\<[27]%
\>[27]{}\mathrel{=}{}\<[27E]%
\>[30]{}\Conid{A}\;(\Varid{listArray}\;(\mathrm{0},\mathrm{2}\mathbin{\uparrow}\mathrm{16}\mathbin{-}\mathrm{1})\;(\Varid{zip}\;(\Varid{elems}\;\Varid{a1})\;(\Varid{elems}\;\Varid{a2}))){}\<[E]%
\\
\>[B]{}\hsindent{5}{}\<[5]%
\>[5]{}\Varid{dom}\;(\Conid{A}\;\Varid{a}){}\<[27]%
\>[27]{}\mathrel{=}{}\<[27E]%
\>[30]{}\Conid{Bag}\;[\mskip1.5mu \Varid{k}\mid (\Varid{k},\Varid{v})\leftarrow \Varid{assocs}\;\Varid{a},\neg \;(\Varid{isNull}\;\Varid{v})\mskip1.5mu]{}\<[E]%
\\
\>[B]{}\hsindent{5}{}\<[5]%
\>[5]{}\Varid{cod}\;(\Conid{A}\;\Varid{a}){}\<[27]%
\>[27]{}\mathrel{=}{}\<[27E]%
\>[30]{}\Conid{Bag}\;[\mskip1.5mu \Varid{v}\mid (\Varid{k},\Varid{v})\leftarrow \Varid{assocs}\;\Varid{a},\neg \;(\Varid{isNull}\;\Varid{v})\mskip1.5mu]{}\<[E]%
\\
\>[B]{}\hsindent{5}{}\<[5]%
\>[5]{}\Varid{lookup}\;(\Conid{A}\;\Varid{a}){}\<[27]%
\>[27]{}\mathrel{=}{}\<[27E]%
\>[30]{}(\mathbin{!})\;\Varid{a}{}\<[E]%
\\
\>[B]{}\hsindent{5}{}\<[5]%
\>[5]{}\Varid{index}\;\Varid{kvps}{}\<[27]%
\>[27]{}\mathrel{=}{}\<[27E]%
\>[30]{}\Conid{A}\;(\Varid{accumArray}\;(\Varid{curry}\;\Varid{\Conid{Bag}.union})\;\Varid{\Conid{Bag}.empty}\;(\mathrm{0},\mathrm{2}\mathbin{\uparrow}\mathrm{16}\mathbin{-}\mathrm{1})\;\Varid{vals}){}\<[E]%
\\
\>[5]{}\hsindent{2}{}\<[7]%
\>[7]{}\mathbf{where}{}\<[E]%
\\
\>[7]{}\hsindent{2}{}\<[9]%
\>[9]{}\Varid{vals}{}\<[27]%
\>[27]{}\mathrel{=}{}\<[27E]%
\>[30]{}(\Varid{elements}\mathbin{\circ}\Varid{fmap}\;(\Varid{\Conid{Bifunctor}.second}\;\Varid{\Conid{Bag}.single}))\;\Varid{kvps}{}\<[E]%
\\
\>[B]{}\hsindent{5}{}\<[5]%
\>[5]{}\Varid{unindex}\;(\Conid{A}\;\Varid{a}){}\<[27]%
\>[27]{}\mathrel{=}{}\<[27E]%
\>[30]{}\Conid{Bag}\;[\mskip1.5mu (\Varid{k},\Varid{v})\mid (\Varid{k},\Varid{vs})\leftarrow \Varid{assocs}\;\Varid{a},\Varid{v}\leftarrow \Varid{elements}\;\Varid{vs}\mskip1.5mu]{}\<[E]%
\ColumnHook
\end{hscode}\resethooks


\noindent
The implementation of the map is simply a constant space array. A brief
explanation of some of the function will be given to give an intuition as to how
the indexed structure works. The $accumArray$ function is often used in the
implementation to create new arrays. The first argument is an accumulation
function that is used only when multiple values are assigned to the same key on
initialisation. Of course this is not necessary for creating an $empty$ map but
it is always set to replace the old value with the new value. Next we assign any
keys with no values to the $null$ element of the $PointedSet$, indicating no
value in this case. We specify the bounds, which is exactly the number of values
a 16 bit word can refer to $2^16 - 1$ and specify any initial key value pairs.
To recap what was said above, for a value to be empty it must have the $null$
value of the $PointedSet$; this helps us also prescribe a domain ($dom$)
function for keys whose value is not $null$ and a codomain ($cod$) function for
values that are not $null$. Merging two $Map$s, a very important operation,
is simply also translated to an underlying array operation where an array is
just created with values in order of the list given (where any empty values are
$null$ and so takes up the whole span of the array). The $index$ function had to
be modified from the original implementation in the paper. In order to allow for
the accumulation function to form a union of bags (as was suggested by the
lambda function in the paper) the value must be prescribed to be of type
$Bag\;a$ in order to avoid a type error and so before the key value pairs can be indexed, all values are
transformed into a singleton $Bag$. I find this to be a reasonable solution as the
preservation of multiplicity is paramount to database systems and that is the
context in which this implementation is being used and so over specifying or
limiting the type does not disrupt the purpose even if it destroys generality; it
was my intention to preserve the sentiment of the lambda in the paper in such a
way that allowed the code to compile and is useful in the context of
benchmarking databases.
