\section{Benchmarking database}
As part of the project a benchmark will be run on the database system outlined
in \fref{chap:database}. As the paper \relalg{} had a focus on their join algorithm, so will the benchmark focus
on the performance of join queries in particular representing the class of query
optimisations most prominent in the paper. This section will describe some design
decisions taken into account when constructing the relations used in the
benchmark and finally describe the \relation{JOINBENCH} relation which will be
used to perform most benchmarks.

\subsection{Initial design decisions}
As explained in \fref{sec:background:dbbenchmarking} there are a number of styles that can be adapted to
conduct benchmarks for database management system implementations. Often the
style is greatly dependent on the context the system works in and the purposes
of the information gathering. It was my decision that a decision support system
had many important parallels to the domain of interest and so to design the
benchmark with a stronger focus on the work that has been conducted in that
area. Following are some characteristics of decision support
systems~\cite{IntroToDatabaseSystems}
that make their structure and benchmarks particularly relevant to this
project.

\subparagraph{Join complexity} Queries in decision support systems typically
have more complicated design this is caused by the need to access many
kinds of facts in order to answer complex queries. Although the
paper \relalg{}~\cite{RelationalAlgebraByWayOfAdjunctions}
does not recommend any advice for dealing with multiple joins on a practical
industrial scale (as the book An introduction to database
systems~\cite{IntroToDatabaseSystems} describes with industry
standard \emph{prejoin}s), its work is central on the efficiency of joins as a
guiding example.

\subparagraph{Ad hoc queries} Decision support frameworks usually rely on ad hoc
queries more than other applications and as such their benchmarks test its
ability to deal with these queries more than other applications~\cite{SetQueryBenchmark, PractitionersIntroduction}. This is a similar
query type that is of interest for this benchmark.

\subparagraph{Integrity unimportant} The paper \relalg{} does not provide any way of
updating the table, only query methods. This is similar to the aspect of
decision support systems who assume that the data is correct and
does not deal with many updates. Therefore little or no emphasis is placed on
testing the integrity of the system. This helps construct a benchmark that more
accurately reflects the algorithmic core of the paper where the focus was on
query optimisation.

\subsection{The \relation{JOINBENCH} relation}\label{sec:benchmark:joinbench}
The \relation{JOINBENCH} relation is an easily synthesised relation designed in
order to help test the performance of join based queries. Its standard set of
attributes allows it to be versatile in testing queries and their specified selectivity.
The relation is entirely comprised of integer attributes in an
attempt to specialise the relation for queries on join optimisations. When
testing joins and selections the key areas of interest are algorithm
optimisations in different situations. The variables we can control when
designing queries are the specificity of the selections or joins and the cardinality
of the two relations in the operation. Therefore a relation with attributes in
the integer domain and readable names makes should make it easier to calculate
query selectivity as well as expected cardinality of operands.

The attributes of the \relation{JOINBENCH} relation can be found in
\fref{tab:joinbenchattributes}. As can be seen, besides the
\relationAttribute{unique} attribute all other attributes are percentage based.

\subparagraph{Percentage based attributes} Percentage based attributes are
important for specifying the selectivity of joins especially. On a high level,
the percentage in the attribute name expresses the expected proportion of the
tuple to be selected by an equality containing only one value in the domain. For
instance a selection using the predicate `$\relationAttribute{twentyPercent} =
0$' is expected to select twenty percent of the relation. Precisely, given a
percentage based attribute of percent $n$, say, the domain of such an integer
would be the numbers $\left\{0, 1, \ldots, \frac{100}{n} - 1\right\}$ (where we
assume that $\frac{100}{n}$ is an integer). An alternative way of thinking about
this is that an $n$-percent attribute partitions the relation such that each
partition has $n$ percent of the tuples.

\subparagraph{The \relationAttribute{unique} attribute} The purpose of the
\relationAttribute{unique} attribute is partly classical and partly pragmatic. In a classical
perspective, primary keys are paramount to relations and so I thought it
imperative to include an attribute that could fill this role. Furthermore,
despite the implementation of relations as bags allowing duplicates (for use in
aggregations) theoretically databases should not allow duplicate tuples and so
giving each tuple a unique identifier is a way to ensure this. More importantly,
the \relationAttribute{unique} attribute allows us to specify selectivity not
based on proportion but rather given numbers. Say we want to select $n$ tuples
(where $n$ is less than the cardinality of the relation), we can simply perform
a selection on unique that takes the first $n$ tuples using the predicate
`$\relationAttribute{unique} < n$'; it is clear the result is a relation with
only $n$ tuples.

\begin{table}[b]
    \centering
    \begin{tabular}{ll}
        \toprule
        Attribute & Domain \\
        \midrule
        \relation{unique} & Int \\
        \relation{onePercent} & Word16 \\
        \relation{twentyPercent} & Word16 \\
        \relation{twentyFivePercent} & Word16 \\
        \relation{fiftyPercent} & Word16 \\
        \relation{evenOnePercent} & Word16 \\
        \relation{oddOnePercent} & Word16 \\
        \bottomrule
    \end{tabular}
    \caption{Attributes and domains (specified as Haskell types) of the
    attributes of the \relation{JOINBENCH} relation.}
    \label{tab:joinbenchattributes}
\end{table}
