\section{Relations}
As part of the project a benchmark will be run on the database system outlined
in \fref{chap:database}. As the paper ``Relational Algebra by way of
Adjunctions'' had a focus on their join algorithm, so will the benchmark focus
on the performance of join queries in particular representing the class of query
optimisations most prominent in the paper. This section will describe some design
decisions taken into account when constructing the relations used in the
benchmark and finally describe the \relation{JOINBENCH} relation which will be
used to perform most benchmarks.

\paragraph{}Following are some characteristics of decision support
systems~\cite{IntroToDatabaseSystems}
that make their structure and benchmarks particularly relevant to this
project.

\subparagraph{Join complexity} Queries in decision support systems typically
have more complicated design this is caused by the need to access many
kinds of facts in order to answer complex queries. Although the
paper Relational Algebra by way of Adjunctions~\cite{RelationalAlgebraByWayOfAdjunctions}
does not recommend any advice for dealing with multiple joins on a practical
industrial scale (as the book An introduction to database
systems~\cite{IntroToDatabaseSystems} describes with industry
standard \emph{prejoin}s), its work is central on the efficiency of joins as a
guiding example.

\subparagraph{Ad hoc queries} Decision support frameworks usually rely on ad hoc
queries more than other applications and as such their benchmarks test its
ability to deal with these queries more than other applications~\cite{SetQueryBenchmark, PractitionersIntroduction}. This is a similar
query type that is of interest for this benchmark.

\subparagraph{Integrity unimportant} The paper Relational Algebra by way of
Adjunctions does not provide any way of
updating the table, only query methods. This is similar to the aspect of
decision support systems who assume that the data is correct and
does not deal with many updates. Therefore little or no emphasis is placed on
testing the integrity of the system. This helps construct a benchmark that more
accurately reflects the algorithmic core of the paper where the focus was on
query optimisation.
