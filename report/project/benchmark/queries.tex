\section{Queries}
This section will present the queries explored during the benchmarking of the
database system. Along with each query, an analysis of its purpose will also be
presented. The benchmark will run each query a number of times against each of
the functions presented in \fref{sec:benchmark:functions} and the mean results
will be analysed and compared during the rest of this chapter. As the method for
describing the queries become increasingly clear, the explanations will become
more concise in this section.

The queries presented all have a similar structure in how they are presented and
form. They are all equijoins, as equijoins are the focus of this project and a
good way to judge the optimisation of a join followed by a selection. At times,
queries also measure the potential of a preceding selection but the functions
are not designed to take this optimisation into consideration. Each join
will be given a name of the form ``join \relationAttribute{x} and
\relationAttribute{y}'' where \relationAttribute{x} and \relationAttribute{y}
are the names of attributes that the equijoin is performed on; the special font
for attributes will be dropped in the query names. It is worth noting that
``join \relationAttribute{x} and \relationAttribute{y}'' is not the same as
``join \relationAttribute{y} and \relationAttribute{x}'' as the order that the
attributes appear in the name is the order they are entered into the function
and this may be significant computationally.

\paragraph{`join onePercent and twentyPercent'} As described above, this query
joins the \relationAttribute{onePercent} and \relationAttribute{twentyPercent}
attributes of the \relation{JOINBENCH} relation. This paragraph will explore
what this means and its significance to the project and benchmark. As described
in \fref{sec:benchmark:joinbench}, the attribute \relationAttribute{onePercent}
has 100 different values all in the range of 0 to 99. Similarly, twentyPercent
has 5 attributes each in the range of 0 to 4. This, in effect, means that any
tuple whose \relationAttribute{onePercent} value is above 4 is discarded. As
\relationAttribute{onePercent} partitions the relation into 100 groups with a
size of 1\% of the relation, we lost 95\% of the tuples in the first table. Of
the remaining 5\% of tuples in the first table, they will each pair with 20\% of
the second table (as the attribute \relationAttribute{twentyPercent} partitions
the second table into 5 groups of 20\% of the relation). For each group, a
Cartesian product is performed with one percent of the total number of tuples
and twenty percent of the total number of tuples. Given $n$ tuples, each group
then has $\frac{n}{100} \cdot \frac{n}{5} = \frac{n^2}{500}$ tuples and as we
worked out above there are 5 groups. Therefore, we expect the result to have
$\frac{n^2}{100}$ tuples. As both tables have $n$ tuples, a Cartesian product
performed on them will have $n^2$ tuples, thus we expect a selectivity
proportion of one percent on all possible pairs. We expect 5 local Cartesian
products when done optimally, all with 0.2\% of the result. To summarise, this
is the equivalent of running a selection with selectivity factor of 1\% on the
first table then a product of 5 equally sized groups.

\paragraph{`join onePercent and onePercent'} This query is symmetric and is an
example that cannot be optimised by an initial selection as the domains of both
attributes are the same. This query joins two partitions of 1\% of the table
together 100 times. Therefore we expect the total tuple count to be $100 \cdot
\frac{n}{100} \cdot \frac{n}{100} = \frac{n^2}{100}$, where $n$ is the number of
tuples in each relation. You will notice that it produces the same number of
tuples as above, but the distinction is in the number and size of local products.
We expect 100 local products all of which create 1\% of the result and 0.01\% of
the total number of possible pairs of tuples.

\paragraph{`join onePercent and fiftyPercent'} This query follows a similar
structure to the previous two, again varying only in number and size of local
Cartesian products. It is worth noting that, similar to `join onePercent and
twentyPercent', this query is not symmetric and the domains do not match. We
expect the total tuple count to be $\frac{n^2}{100}$ when the
\relation{JOINBENCH} relation has $n$ tuples. There are only two possible groups
that match for the equijoin and every tuple that matches with
\relationAttribute{onePercent} will be paired with around half of all tuples in
the second table, making a substantially large difference in tuple count.

\paragraph{`join twentyPercent and onePercent'} This query exploits the fact
that the orders the attributes are given to the equijoin matters computationally
and is present to test the difference in performance depending on the order. All
the other statistics are the same as the query `join onePercent and
twentyPercent'.

\paragraph{'join evenOnePercent and oddOnePercent'} This query is drastically
different to the others discussed and tests an edge case of selectivity. As
evenOnePercent and oddOnePercent only have even and odd numbers respectively,
the intersection of their domains is empty and therefore the resulting table is
also expected to be empty. This query is present to test the performance of the
different functions under such a bizarre and edge condition to see which can do
the least pointless computation.
