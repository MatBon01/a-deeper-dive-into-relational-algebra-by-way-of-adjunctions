\chapter{Evaluation}
\begin{comment}
Be warned that many projects fall down through poor evaluation. Simply building a system and documenting its design and functionality is not enough to gain top marks. It is extremely important that you evaluate what you have done both in absolute terms and in comparison with the state of the art. This might involve quantitative evaluation, for example based on numerical results, performance etc. or something more qualitative such as expressibility, functionality, ease-of-use etc., possible with respect to a target user base. The evaluation, often coupled with a discussion of future work (see below), should make clear the strengths and weaknesses of what you have done. It is important to understand that there is no such thing as a perfect project. Even the very best pieces of work have their limitations, so you are expected to provide a proper critical appraisal.
\end{comment}
\section{Database implementation} % Haskell database implementation
\paragraph{Correctness} The most important quality for the database system to
have is correctness as if queries are not producing expected results, it does
not matter how long it takes. My strategy for ensuring correctness is rigorous
unit tests. By automatically checking that different modules of the system reach
the same result I do by hand helps ensure that it is working as expected. I
acknowledge that this is not complete however and integration tests are vital;
at times I print the result of large computations and check that they are close
to what I would expect but I doubt I have full integration coverage with
heuristics like this. Given more time it would be very beneficial to use a
preexisting DBMS to run the same queries and compare outputs. This could not be
done now due to the extra tooling that would be required for the automatic
parsing and comparisons needed to take place. Some of the suggestions in
\fref{sec:evaluation:syntheticdatabase} may be useful to help automatically
program parsers and interface code to allow this to happen more easily, but as
mentioned in the section it would require the programs to have a larger idea of
the context of its actions. This is especially true when thinkingi about
queries, it is likely that the queries would need to be transcribed to SQL by
hand, adding in an extra error-prone step. To conclude, I am confident in the
results due to extensive unit testing but I strongly believe that in future an
effort should be made to include both integration tests for functions used in
benchmarking and system tests by comparing results with an existing DBMS.

\paragraph{Readability and efficiency} It is clear that much of the code
presented in \fref{chap:database} could be made more efficient and
readable. The effects of this have varying degrees of severity. To begin with a
more high severity approach the efficiency of some parts of the code present
could reasonably become hot spots that bias results of the benchmarking.
Luckily, in the preliminary profiling undertaken of the code it did not seem to
be an issue, the algorithm for equality of bags could easily bias results to
functions that compare more often and frame potentially viable solutions as much
worse than they are. Of course the asymptotic complexity of the algorithm is
important to consider, the project also had a focus on trends and needlessly
slow algorithms may alter intersection points. In addition, readability is useful for further extending the
database system to approach a more real world use case. Code that
is written more nicely, especially important in the context of academic work,
makes it easier to understand and express the theory behind.
 % Haskell database implementation
\section{Synthetic database creation}
This section evaluates the design decisions and outcomes in creating a synthetic
database generator as outlined in \fref{sec:benchmark:syntheticdatabase}. This
project introduced a framework to generate database to fit a custom schema. The
benefits of synthetic data are vast and include the ability to define schemas
that are specialised in benchmarking and readability while controlling frequency
of values and other important properties.

\paragraph{Comparison to existing solutions} Overall the result was quite different from other technologies available,
offering a low-level bare-bones solution. Many existing solutions equally support
qualitative data to quantitative with selling points based on the
\lstinline{Faker} python module \cite{Faker}. Although \lstinline{Faker} did not come to my
attention while designing the framework, it would not have been much use for the
\database{JOINBENCH} database in its final form as well. Additionally, much
infrastructure would still need to be made around the \lstinline{Faker} module
for easy exporting and specification, therefore it would not have saved much
time and allowed me to develop a more customised solution for the problem.
Advantages of the \lstinline{Faker} framework over my solution is the vast
amount of data available for attributes that were considered to be qualitative,
although there would still be time necessary in order to do due diligence on its
copyright and licensing creating further ethical considerations. The \lstinline{Faker} module has made interesting
design decisions, favouring a \lstinline{unique} attribute to composition for
uniqueness which itself comes with benefits and drawbacks. The \lstinline{Faker}
module does present interesting opportunities to enhance my solution however:
Using an adapter design pattern a faker object could easily be converted into
using the cell API and enrich my solution with a larger variety of data,
although I would accept the argument that it might be a waste of the set of rich
features implemented by the module. Using the \lstinline{Faker} module implicitly
with other data synthesising modules that depend on
it, such as \lstinline{pydbgen} \cite{pydbgen}, can give a more complete solution even in
comparison to my framework. In essence, it uses the variety of data sources
provided by framework compiled into a variety of different outputs formats.

Moreover, other solutions with very different aims exist. Academics in the US
have developed DataSynthesizer \cite{DataSynthesizer} which is a privacy
oriented synthetic data generator for collaborating with sensitive data. Given
an input of a sensitive data the technology comes in three parts, a
\lstinline{DataDescriber} which analyses the data and describes its form, a
\lstinline{DataGenerator} which takes this information and creates a
predetermined number of rows, and finally a \lstinline{ModelInspector} for
determining the similarity of the produced data to the private data. It is clear
in order to use the technology as intended I would already require a dataset to
reproduce. Hypothetically, singling out the \lstinline{DataGenerator} and
providing a hand-written description may also not be a very specialised
solution. Despite the fact that it is not the intended purpose of the system,
the generation is clearly created to be able to represent real world data, which
the \relation{JOINBENCH} relation is far from. Therefore, other more specialised
solutions may be more useful.

In conclusion, despite the rich variety of tools already available for the
problem domain I believe the infrastructure created during this project for
synthetic database creation is a useful contribution. Although, when taken for
their intended purpose most alternate solution provide a more complete and
beneficial solution, I believe the framework created by me is low level enough
to be manipulatable in a clean way to express a wide range of properties that
might be needed for a benchmark. Benchmarks are characterised by the need to
strictly control the data and express properties on the data easily and
specifically, this may suggest that a low level solution may be better suited
and when organised as such may result in cleaner code than trying to adapt
higher level solutions. Although, ``low-level'' and ``bare-bones''
might sound like drawbacks of deliverable it was intentionally designed and serves
as a different style of implementation to the other solutions available; a style
I believe suits the intricacies of the project and leaves much room for
extensions if necessary in future.


\paragraph{Evaluation of overall design and potential future work} The framework
was designed from a pragmatic point of view for the task at hand; as such,
despite my best efforts to keep the tool general and open for future use, some
limitations in the scope currently developed have risen. My concerns lie in two
main categories, interactions and overall system knowledge by the program.

\subparagraph{Interactions} Especially as cell composition became more prevalent
in the design or other potential dependencies between cells arose the potential
untidiness of the solution lurked in. An example currently in the system is the
way \lstinline{RecordCell} strays from the simple \lstinline{Cell} API. Although
the solution prides itself in being low level and customisable by code, it
feels like a stray away from a uniformity that in future may be exploited for
different spin off solutions as may be described later in this section. Other
instances of when extra code may need to be introduced due to unforeseen
dependencies between components may be cells that require a larger context to
function such as
the total number of records to be generated. In current implementations, only
higher levels of the system are made aware of this, also at the last possible
moment. Again, these are not large issues but concerns of difficulties as
use cases may scale; a client could simply initialise the needed
\lstinline{Cell}s with the needed context at runtime. Therefore this may only
be an obstacle for potential automation and better message passing mechanisms
might need to be developed between client and framework or within the framework
itself to allow for cleaner solutions.

\subparagraph{System knowledge} The framework currently is lacking in knowledge
of the system larger than what it was primarily created for, the generation of
data. In other words, the program is not aware of the context it is running in,
such as the schema of the database as a whole. Many extensions could be created
by giving the program more knowledge, unnecessary for just the synthesis of
data.

In the current implementation the data generated comes entirely from the
\lstinline{RecordGenerator} and larger structures defined by the user. Notably,
this lacks a larger knowledge, such as the exact domain of the attributes the
cells are working in and attribute names. The return type of \lstinline{Cell}'s
\lstinline{generate} function is \lstinline[language=Python]{str} as it is what
it is needed to output the data into a plaintext CSV file. If design changes
were made to give the system a greater knowledge of the larger context and
environment it is contributing to it can generate more useful structures; an
example that would benefit this project would be automatically generating
Haskell parsers to read the database into memory. Another more general example of when
knowledge of the larger context may be useful is in automatic generation of SQL
statements to create the database that output methods then might be able to
populate during data synthesis. On the other hand, if implemented incorrectly,
requiring a greater description of the system in order to generate data obscures
its initial purpose and adds more bloat when using the software; the cell class
should not need to know that the attribute is called ``evenOnePercent'' in order
to generate even numbers.

\paragraph{Component evaluation} We now briefly evaluate the implementations of
components within the framework as some interesting challenges and limitations
appeared while designing the system.

\subparagraph{Cells} Cells are the most important component of the system design
wise and, in my opinion, introduce the largest surface for design difficulties.
Even, as mentioned in the previous paragraph, deciding the return types and
number of abstract methods in the class can either introduce a whole level of
extremely useful functionality or bloat the system. This part of the report will
evaluate some more difficult design decisions made in cells and cell examples.

Composition of cells was introduced in the project as a way of adding another
level of complexity to the decisions developers were able to make. The examples
coded during the project, however, tended to be quite inefficient and general;
they would definitely need to be redefined and optimised to work at scale.
Despite the early inefficient implementations of composition the generality they
must improve is a systematic problem prescribed by the type of solution that may result in unreadable, yet
efficient, solutions if taken seriously in further development. The
generality described here is found in the inability to apply different
algorithms when different properties are given in the domain. Going back to the
example in \fref{sec:benchmark:syntheticdatabase} for cell composition,
\lstinline{unique_nums}. The algorithm used in \lstinline{unique_nums}, which
to remind you is simply a \lstinline{RandomModularIntegerCell} in a
\lstinline{UniqueCell} is quite inefficient and generates new values until an unseen
integer is found. It has been widely known for decades that algorithms to
produce a random permutations of integers are prevalent and a large discussion
of methods to generate dense
unique random data can be found in Gray's paper ``Quickly Generation
Billion-Record Synthetic Databases'' \cite{GenLargeSynthDB}. This shows that
many more suitable ways of generating unique dense integers exist and open the
question as to other optimisations that may be made to \lstinline{UniqueCell}
for other more specialised areas. A potential solution may be to define a range
of specialised cells (and not using composition in these cases) as problems
arise. I find this solution to be acceptable as I can personally justify new
classes for unique cases. It is clear, however, that the number of cells may
grow greatly as the framework is extended in this way and the correct times to
use cell composition may become more confusing; it is
clearly a trade off between optimisation and readability. I believe that this
issue also highlights the strength my solution has in extensibility as the
question is not whether the framework allows such structures, but whether we
should and users of this system can make their own decisions and implement such
cells for themselves and their use cases.
 % Synthetic database creation
\section{Benchmark methodology and results}
\subsection{\relation{JOINBENCH} relation and queries}
The \relation{JOINBENCH} relation is a fundamental step in the design of
the benchmark as it singlehandedly determines the possible complexity of the
possible queries. In this regard I think the \relation{JOINBENCH} was mostly
sufficient, it was largely based on previously designed benchmarks with slight
modifications to make it more specialised for the type of interesting
selection and join based queries that would be interesting in this project. One
attribute and family of queries that I found to be missing during my analysis,
however, revolved around the idea of letting the local Cartesian products grow
to be the same size as a normal product on the relations. This could be solved
by adding an attribute \relationAttribute{oneHundredPercent} whose domain only
contained one value (thus containg 100\% of the relation). I believe that this
would more interestingly highlight the disadvantages of the indexed approach as
it is likely to be reduced to a normal product. Admittedly I speculate that the lack of a need
for a filter on $n^2$ elements would still make the indexed equijoin dominant
over the product equijoin; however, I believe it would be much closer to the
comprehension equijoin. Alternatively, this could be done by conducting a
selection on an existing attribute and then joining that attribute to itself,
but I believe this goes against the design decisions expressed in
\fref{sec:background:benchmarkbestpractices}. Taken to the extreme, all
attributes in the relation could be created by various selections on the
\relationAttribute{onePercent} but this not only makes standardising relation
cardinality more difficult, it destroys the readability synthetic data sets are
meant to express. It may be interesting to include attributes in future that
cannot be derived from \relationAttribute{onePercent}, for instance other
multiples such as \relationAttribute{oneThird}, but I am not sure how much more
information this would provide in this case. These suggestion may be taken
forward as future work on designing the \relation{JOINBENCH} relation.

On the topic of readability of attributes, I feel that the naming scheme has
mixed success. The attributes were names after similar attributes in the updated
Wisconsin relation but were kept due to the importance of the cardinality of
partitions to this domain. However, I do find that the attributes names struggle
to easily convey the values in their domains somewhat limiting the expression of
the database. I do feel, however, that if the name gave more information about
the domain it would conversely be more difficult to figure out the partition
sizes. Similarly, all queries are based on uniform sizing and partitions. It
would be a very interesting piece of future work to see how dynamic sized local
joins impact indexing; we could ask questions such as whether a joins with one
partition with a larger cardinality and few other small ones may be faster than
a join with relations with larger tuple counts but many small local partitions.
Having non uniform sizes in the queries would make the benchmark more realistic
to real world scenarios but I also struggle to see a scientific question that
may be asked. Furthermore, I feel the theoretical explanations of the results
obtained in this experiment as described in \fref{sec:benchmark:results}
already gives enough understanding to extrapolate and model into the
non-standard cases.

\subsection{Results}
This subsection will evaluate the results presented in
\fref{sec:benchmark:results} and the processes that obtained them.
At first glance the results of the benchmark seem to be very reasonable and
trends are smooth, expected and clear. A more thorough analysis can be completed
by checking the standard deviations associated with the results. Tables with the
mean and standard deviation for a select number of queries can be found in
\fref{tab:evaluation:std-dev-comparison-onePercent-onePercent},
\fref{tab:evaluation:std-dev-comparison-onePercent-fiftyPercent} and
\fref{tab:evaluation:std-dev-comparison-evenOnePercent-oddOnePercent}. I have
chosen these queries as I believe they most accurately sample the space of
queries and edge cases. I have included the means in this result to give a
comparison of order between the results and spread. It is clear that the
standard deviation in most cases are 3 orders of magnitude smaller than the
mean, potentially indicating that the `quiet' computational environment
(described in \fref{sec:benchmark:experiment}) the
results were gathered in was successful in producing reproducible data. It is
worth noting that the function indexed equijoin continuously seems to have an
order of magnitude higher standard deviation, suggesting a larger spread of
data. This could be due to the fact that the function is more computationally
complex and therefore more sensitive the runtime environment. Most importantly,
due to the nature of such a small spread, it is unlikely that results wrong
relative to each other (no lines overlap) and therefore the conclusions drawn
may be more confidently accepted despite any statistical noise and uncertainty.

\begin{table}[h]
\centering
\begin{tabular}{llll}
\toprule
 & 1000 tuples & 5000 tuples & 10000 tuples \\
\midrule
Product equijoin & 0.0174 ± 7.14e-05 & 0.436 ± 4.54e-05 & 1.72 ± 0.0025 \\
Comprehension equijoin & 0.00848 ± 7.5e-06 & 0.209 ± 7.66e-05 & 0.835 ± 0.000143 \\
Indexed equijoin & 0.00457 ± 5.17e-05 & 0.0242 ± 0.000301 & 0.0855 ± 0.00477 \\
\bottomrule
\end{tabular}

\caption{A table showing the mean time (s) and standard deviation (s) to complete the query `join onePercent and onePercent' for each function.}
\label{tab:evaluation:std-dev-comparison-onePercent-onePercent}
\end{table}

\begin{table}[h]
\centering
\begin{tabular}{llll}
\toprule
 & 1000 tuples & 5000 tuples & 10000 tuples \\
\midrule
Product equijoin & 0.0197 ± 4.09e-05 & 0.49 ± 0.000208 & 1.94 ± 0.00101 \\
Comprehension equijoin & 0.00853 ± 4.38e-06 & 0.217 ± 0.00233 & 0.882 ± 0.00256 \\
Indexed equijoin & 0.0077 ± 0.000159 & 0.0724 ± 0.0017 & 0.358 ± 0.00933 \\
\bottomrule
\end{tabular}

\caption{A table showing the mean time (s) and standard deviation (s) to
complete the query `join onePercent and fiftyPercent' for each function.}
\label{tab:evaluation:std-dev-comparison-onePercent-fiftyPercent}
\end{table}

\begin{table}[h]
\centering
\begin{tabular}{llll}
\toprule
 & 1000 tuples & 5000 tuples & 10000 tuples \\
\midrule
Product equijoin & 0.019 ± 1.71e-05 & 0.477 ± 8.97e-05 & 1.87 ± 0.000181 \\
Comprehension equijoin & 0.00811 ± 2.22e-06 & 0.202 ± 3.31e-05 & 0.791 ± 0.000797 \\
Indexed equijoin & 0.00358 ± 2.23e-05 & 0.00743 ± 6.81e-05 & 0.0153 ± 0.000118 \\
\bottomrule
\end{tabular}

\caption{A table showing the mean time (s) and standard deviation (s) to
complete the query `join evenOnePercent and oddOnePercent' for each function.}
\label{tab:evaluation:std-dev-comparison-evenOnePercent-oddOnePercent}
\end{table}

On the other hand, despite the results having a contained spread, the
conclusions drawn in \fref{sec:benchmark:results} have much space to become more
rigorous. Firstly, a hypothesis test may be conducted to determine whether the
claims are significant in a more prescribed way and given more time I think this
is an important next step. Secondly, much of the section was based on comments
of trends and despite the data clearly being consistent within repeats of the
same query no statistical analysis has been conducted on the trends. Because of
the clear visualisations of the data I am confident that the trends in question
are existent however it would be vital to conduct an analysis to model the
relationships between tuple counts, different queries and functions. This is
clearly where the results fall short and future work would need to be done in
order to add rigour to the already clear patterns the data shows.

Finally, another shortcoming of the results is the potential improvements that
could have been made to statistics considered and ways in which the tools in
question were understood and used. The mean average can easily be swayed by
outliers and, although the spread suggests the data was quite tight and
outliers not too extreme, this could have been a heavy issue. Computer
benchmarking can be prone to outliars for a variety of factors from noise on the
system resources to caching effects and therefore more care should have been
taken to mitigate these risks. However a quick heuristic analysis of the median
and means shows that in this case they are not far apart. Moreover, the
\verb|Criterion| library itself gives a number of more reliable statistics that
I did not use in this project because of time and scope constraints. These may
have led to more reliable results. More time could have been spent on modifying
default options for the library to ensure that sampling rates and other useful
variables were consistent and the benchmarks done in a much fairer way. Luckily,
I believe these did not have a serious impact on the results as the data
gathered seems to be consistent and patterns significantly distinct and clear. I
believe it would also be important to report the usage of other computer
resources, especially CPU utilisation and memory, as these are important
considerations when choosing a query function and standard practice in the academic
database benchmarking community as discussed in
\fref{sec:background:benchmarkbestpractices}.

 % Benchmark methodology and results
