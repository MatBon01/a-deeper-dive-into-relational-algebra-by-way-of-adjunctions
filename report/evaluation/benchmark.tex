\section{Benchmark methodology and results}
\subsection{\relation{JOINBENCH} relation and queries}
The \relation{JOINBENCH} relation is a fundamental step in the design of
the benchmark as it singlehandedly determines the possible complexity of the
possible queries. In this regard I think the \relation{JOINBENCH} was mostly
sufficient, it was largely based on previously designed benchmarks with slight
modifications to make it more specialised for the type of interesting
selection and join based queries that would be interesting in this project. One
attribute and family of queries that I found to be missing during my analysis,
however, revolved around the idea of letting the local Cartesian products grow
to be the same size as a normal product on the relations. This could be solved
by adding an attribute \relationAttribute{oneHundredPercent} whose domain only
contained one value (thus containg 100\% of the relation). I believe that this
would more interestingly highlight the disadvantages of the indexed approach as
it is likely to be reduced to a normal product. Admittedly I speculate that the lack of a need
for a filter on $n^2$ elements would still make the indexed equijoin dominant
over the product equijoin; however, I believe it would be much closer to the
comprehension equijoin. Alternatively, this could be done by conducting a
selection on an existing attribute and then joining that attribute to itself,
but I believe this goes against the design decisions expressed in
\fref{sec:background:benchmarkbestpractices}. Taken to the extreme, all
attributes in the relation could be created by various selections on the
\relationAttribute{onePercent} but this not only makes standardising relation
cardinality more difficult, it destroys the readability synthetic data sets are
meant to express. It may be interesting to include attributes in future that
cannot be derived from \relationAttribute{onePercent}, for instance other
multiples such as \relationAttribute{oneThird}, but I am not sure how much more
information this would provide in this case. These suggestion may be taken
forward as future work on designing the \relation{JOINBENCH} relation.

On the topic of readability of attributes, I feel that the naming scheme has
mixed success. The attributes were names after similar attributes in the updated
Wisconsin relation but were kept due to the importance of the cardinality of
partitions to this domain. However, I do find that the attributes names struggle
to easily convey the values in their domains somewhat limiting the expression of
the database. I do feel, however, that if the name gave more information about
the domain it would conversely be more difficult to figure out the partition
sizes. Similarly, all queries are based on uniform sizing and partitions. It
would be a very interesting piece of future work to see how dynamic sized local
joins impact indexing; we could ask questions such as whether a joins with one
partition with a larger cardinality and few other small ones may be faster than
a join with relations with larger tuple counts but many small local partitions.
Having non uniform sizes in the queries would make the benchmark more realistic
to real world scenarios but I also struggle to see a scientific question that
may be asked. Furthermore, I feel the theoretical explanations of the results
obtained in this experiment as described in \fref{sec:benchmark:results}
already gives enough understanding to extrapolate and model into the
non-standard cases.

\subsection{Results}
This subsection will evaluate the results presented in
\fref{sec:benchmark:results} and the processes that obtained them.
At first glance the results of the benchmark seem to be very reasonable and
trends are smooth, expected and clear. A more thorough analysis can be completed
by checking the standard deviations associated with the results. Tables with the
mean and standard deviation for a select number of queries can be found in
\fref{tab:evaluation:std-dev-comparison-onePercent-onePercent},
\fref{tab:evaluation:std-dev-comparison-onePercent-fiftyPercent} and
\fref{tab:evaluation:std-dev-comparison-evenOnePercent-oddOnePercent}. I have
chosen these queries as I believe they most accurately sample the space of
queries and edge cases. I have included the means in this result to give a
comparison of order between the results and spread. It is clear that the
standard deviation in most cases are 3 orders of magnitude smaller than the
mean, potentially indicating that the `quiet' computational environment
(described in \fref{sec:benchmark:experiment}) the
results were gathered in was successful in producing reproducible data. It is
worth noting that the function indexed equijoin continuously seems to have an
order of magnitude higher standard deviation, suggesting a larger spread of
data. This could be due to the fact that the function is more computationally
complex and therefore more sensitive the runtime environment. Most importantly,
due to the nature of such a small spread, it is unlikely that results wrong
relative to each other (no lines overlap) and therefore the conclusions drawn
may be more confidently accepted despite any statistical noise and uncertainty.

\begin{table}[h]
\centering
\begin{tabular}{llll}
\toprule
 & 1000 tuples & 5000 tuples & 10000 tuples \\
\midrule
Product equijoin & 0.0174 ± 7.14e-05 & 0.436 ± 4.54e-05 & 1.72 ± 0.0025 \\
Comprehension equijoin & 0.00848 ± 7.5e-06 & 0.209 ± 7.66e-05 & 0.835 ± 0.000143 \\
Indexed equijoin & 0.00457 ± 5.17e-05 & 0.0242 ± 0.000301 & 0.0855 ± 0.00477 \\
\bottomrule
\end{tabular}

\caption{A table showing the mean time (s) and standard deviation (s) to complete the query `join onePercent and onePercent' for each function.}
\label{tab:evaluation:std-dev-comparison-onePercent-onePercent}
\end{table}

\begin{table}[h]
\centering
\begin{tabular}{llll}
\toprule
 & 1000 tuples & 5000 tuples & 10000 tuples \\
\midrule
Product equijoin & 0.0197 ± 4.09e-05 & 0.49 ± 0.000208 & 1.94 ± 0.00101 \\
Comprehension equijoin & 0.00853 ± 4.38e-06 & 0.217 ± 0.00233 & 0.882 ± 0.00256 \\
Indexed equijoin & 0.0077 ± 0.000159 & 0.0724 ± 0.0017 & 0.358 ± 0.00933 \\
\bottomrule
\end{tabular}

\caption{A table showing the mean time (s) and standard deviation (s) to
complete the query `join onePercent and fiftyPercent' for each function.}
\label{tab:evaluation:std-dev-comparison-onePercent-fiftyPercent}
\end{table}

\begin{table}[h]
\centering
\begin{tabular}{llll}
\toprule
 & 1000 tuples & 5000 tuples & 10000 tuples \\
\midrule
Product equijoin & 0.019 ± 1.71e-05 & 0.477 ± 8.97e-05 & 1.87 ± 0.000181 \\
Comprehension equijoin & 0.00811 ± 2.22e-06 & 0.202 ± 3.31e-05 & 0.791 ± 0.000797 \\
Indexed equijoin & 0.00358 ± 2.23e-05 & 0.00743 ± 6.81e-05 & 0.0153 ± 0.000118 \\
\bottomrule
\end{tabular}

\caption{A table showing the mean time (s) and standard deviation (s) to
complete the query `join evenOnePercent and oddOnePercent' for each function.}
\label{tab:evaluation:std-dev-comparison-evenOnePercent-oddOnePercent}
\end{table}

On the other hand, despite the results having a contained spread, the
conclusions drawn in \fref{sec:benchmark:results} have much space to become more
rigorous. Firstly, a hypothesis test may be conducted to determine whether the
claims are significant in a more prescribed way and given more time I think this
is an important next step. Secondly, much of the section was based on comments
of trends and despite the data clearly being consistent within repeats of the
same query no statistical analysis has been conducted on the trends. Because of
the clear visualisations of the data I am confident that the trends in question
are existent however it would be vital to conduct an analysis to model the
relationships between tuple counts, different queries and functions. This is
clearly where the results fall short and future work would need to be done in
order to add rigour to the already clear patterns the data shows.

