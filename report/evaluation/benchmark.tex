\section{Benchmark methodology and results}
\subsection{\relation{JOINBENCH} relation and queries}
The \relation{JOINBENCH} relation is a fundamental step in the design of
the benchmark as it singlehandedly determines the possible complexity of the
possible queries. In this regard I think the \relation{JOINBENCH} was mostly
sufficient, it was largely based on previously designed benchmarks with slight
modifications to make it more specialised for the type of interesting
selection and join based queries that would be interesting in this project. One
attribute and family of queries that I found to be missing during my analysis,
however, revolved around the idea of letting the local Cartesian products grow
to be the same size as a normal product on the relations. This could be solved
by adding an attribute \relationAttribute{oneHundredPercent} whose domain only
contained one value (thus containg 100\% of the relation). I believe that this
would more interestingly highlight the disadvantages of the indexed approach as
it is likely to be reduced to a normal product. Admittedly I speculate that the lack of a need
for a filter on $n^2$ elements would still make the indexed equijoin dominant
over the product equijoin; however, I believe it would be much closer to the
comprehension equijoin. Alternatively, this could be done by conducting a
selection on an existing attribute and then joining that attribute to itself,
but I believe this goes against the design decisions expressed in
\fref{sec:background:benchmarkbestpractices}. Taken to the extreme, all
attributes in the relation could be created by various selections on the
\relationAttribute{onePercent} but this not only makes standardising relation
cardinality more difficult, it destroys the readability synthetic data sets are
meant to express. It may be interesting to include attributes in future that
cannot be derived from \relationAttribute{onePercent}, for instance other
multiples such as \relationAttribute{oneThird}, but I am not sure how much more
information this would provide in this case. These suggestion may be taken
forward as future work on designing the \relation{JOINBENCH} relation.

\subsection{Results}
