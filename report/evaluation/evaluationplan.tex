\chapter{Evaluation plan} % 1-2 pages
\begin{comment}
Project evaluation is very important, so it's important to think now about how you plan to measure success. For example, what functionality do you need to demonstrate?  What experiments to you need to undertake and what outcome(s) would constitute success?  What benchmarks should you use? How has your project extended the state of the art?  How do you measure qualitative aspects, such as ease of use?  These are the sort of questions that your project evaluation should address; this section should outline your plan.
\end{comment}
Evaluation of this project is important as the analyses need to be founded in order to have any sort of real benefit.

We split the evaluation of the implementation into two halves, the evaluation of the correctness and the evaluation of the analysis.

\section{Correctness}\label{sec:evaluationduringimplementation}
In order to evaluate the correctness it is vital to keep good coding practice throughout the project. It is not possible to analyse the performance of something incorrectly implemented, to justify the benefits of a different system entirely.

In order to do this, unit tests should be written during production, to ensure no developer error or mistake occurs by accident.

In order to further reduce mistakes, both bugs and theoretical misunderstandings if there are any the results of the two equijoin operations should be compared to ensure that the results are the same and thus they are being judged on the query interpreted in the same way.