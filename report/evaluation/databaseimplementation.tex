\section{Database implementation} % Haskell database implementation
\paragraph{Correctness} The most important quality for the database system to
have is correctness as if queries are not producing expected results, it does
not matter how long it takes. My strategy for ensuring correctness is rigorous
unit tests. By automatically checking that different modules of the system reach
the same result I do by hand helps ensure that it is working as expected. I
acknowledge that this is not a complete solution, however, and integration tests are vital;
at times, I print the result of large computations and check that they are close
to what I would expect but I doubt I have full integration coverage with
heuristics like this. Given more time, it would be very beneficial to use a
pre-existing DBMS to run the same queries and compare outputs. This could not be
done now due to the extra tooling that would be required for the automatic
parsing and comparisons needed to take place. Some of the suggestions in
\fref{sec:evaluation:syntheticdatabase} may be useful to help automatically
program parsers and interface code to allow this to happen more easily, but as
mentioned in the section it would require the programs to have a larger idea of
the context of its actions. This is especially true when thinking about
queries, it is likely that the queries would need to be transcribed to SQL by
hand, adding in an extra error-prone step. To conclude, I am confident in the
results due to extensive unit testing but I strongly believe that in the future an
effort should be made to include both integration tests for functions used in
benchmarking and system tests by comparing results with an existing DBMS.

\paragraph{Comprehension notation} Much of the paper \relalg{} emphasised the
usefulness of comprehension notation for programmers and indicated this as a
motivating factor to develop the theory of joins in such a way. Although in the
paper comprehension notation was suggested, in the project a mixture of scope
and limitations I believe to be with GHC prevented me from being able to use
monad comprehensions combined with SQL transformation notation to combine two
bags using indexed strategies. This is a large limitation of the project as it
would have been nice to see the spirit of the original paper in the results.
Furthermore, we already saw major benefits to using `comprehension equijoin'
over `product equijoin' and it would be interesting to see if there is a similar
comparison of efficiency between `indexed equijoin' and whatever inhomogeneous
monad comprehension would allow that to happen. However, I feel that the results
in the paper show more than enough about the superiority and limitations of the
`indexed equijoin' and if any comprehension would be syntactic sugar for such a
method (perhaps with some compiler optimisations) similar patterns should be
expected.

\paragraph{Readability and efficiency} It is clear that much of the code
presented in \fref{chap:database} could be made more efficient and
readable. The effects of this have varying degrees of severity. To begin with a
more high-severity approach, the efficiency of some parts of the code present
could reasonably become hot spots that bias results of the benchmarking.
Luckily, in the preliminary profiling undertaken of the code it did not seem to
be an issue, the algorithm for equality of bags could easily bias results
towards
functions that compare more often and frame potentially viable solutions as much
worse than they are. Of course the asymptotic complexity of the algorithm is
important to consider, the project also had a focus on trends and needlessly
slow algorithms may alter intersection points. In addition, readability is useful for further extending the
database system to approach a more real-world use case. Code that
is written more nicely, especially important in the context of academic work,
makes it easier to understand and express the theory behind.
