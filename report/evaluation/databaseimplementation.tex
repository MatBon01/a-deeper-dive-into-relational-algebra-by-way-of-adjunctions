\section{Database implementation} % Haskell database implementation
\paragraph{Correctness} The most important quality for the database system to
have is correctness as if queries are not producing expected results, it does
not matter how long it takes. My strategy for ensuring correctness is rigorous
unit tests. By automatically checking that different modules of the system reach
the same result I do by hand helps ensure that it is working as expected. I
acknowledge that this is not complete however and integration tests are vital;
at times I print the result of large computations and check that they are close
to what I would expect but I doubt I have full integration coverage with
heuristics like this. Given more time it would be very beneficial to use a
preexisting DBMS to run the same queries and compare outputs. This could not be
done now due to the extra tooling that would be required for the automatic
parsing and comparisons needed to take place. Some of the suggestions in
\fref{sec:evaluation:syntheticdatabase} may be useful to help automatically
program parsers and interface code to allow this to happen more easily, but as
mentioned in the section it would require the programs to have a larger idea of
the context of its actions. This is especially true when thinkingi about
queries, it is likely that the queries would need to be transcribed to SQL by
hand, adding in an extra error-prone step. To conclude, I am confident in the
results due to extensive unit testing but I strongly believe that in future an
effort should be made to include both integration tests for functions used in
benchmarking and system tests by comparing results with an existing DBMS.
