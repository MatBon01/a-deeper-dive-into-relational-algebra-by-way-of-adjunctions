The project itself (in its early stages) does not have a novel ideal as a useable output and thus is very neutral in its ethical impact. Instead it has the ability to recommend or withhold recommendation of an earlier approach to database optimisations via implementation and benchmarking results, and in its final stages may recommend further optimisations and so the more interesting ethical discussion is around the ethics of database optimisations.


It is worth noting that the early stages of the project involves implementing the ideas created in \cite{RelationalAlgebraByWayOfAdjunctions}. This causes no legal issues as there is no protection on an idea, lacking the tangibility (and ability to be patented) \todo{See exact laws}. On an ethical level, despite implementations, improvements and building on work is commonplace, encouraged and vital to the academic community, one of the main contributors of the paper is my supervisor and I interpret this as implicit consent. With good referencing practice, plagiarism would also be completely avoided.

There is an interesting discussion to be had on the ethical impact of the benchmarking dataset to be used in the early stages of the project. In order to evaluate the suggestions in \cite{RelationalAlgebraByWayOfAdjunctions} a database system will need to be implemented, but more consequentially one or more data sources will need to be used to evaluate the queries. I do not think that creating a novel data source will be very useful or worthwhile so I will need to acquire at least one somehow, raising at least two issues.
\subparagraph*{Data protection} The immediate issue that springs to mind would be whether the data source contains personal data and if so the subsequent data protection methods required to fulfil my legal and ethical obligation. To avoid this issue for primary data sources, I will not collect data when creating a benchmarking database, instead opting to randomly generate records. For data sources I acquire, they will all be in public domain already, and avoid sensitive personal data. This should minimise the risk to individuals affected. The data protection laws consider "organisation" and "structuring" as processing personal data, hence why this is relevant. The databases acquired might need to be reorganised into the data structured explained in this report \todo{cite}. \todo{See if this is enough} \todo{Cite data protection laws}
\subparagraph*{Database rights} Another consideration when outsourcing database collections, is the extent to which those who have created the database are protected. The creation of a database, although not creative, has been recognised as taking significant work and so those that create the database have a form of copyright protection on their work. \todo{Make sure that this is accurate, cite if so} This risk will be mitigated by ensuring that all databases used are in the public domain and the correct licenses are acquired.

\todo{Potentially comment on the distribution of my implementation}

\section{Extensions}
Furthermore, any domains investigated in this project have their own set of ethical considerations which reasonably cannot be discussed now (as they are largely unknown). Care should be taken on thinking of the ethical impact of the work that is being taken when branching out to other scenarios that use bulk-types. For instance, the suggested family of languages -- logic programming -- has close ties with artificial intelligence which itself is an ethically complex subject. \todo{Cite AI ethics} I think it is fair to say, however, that the work done will not bias any field in a particular way (ethical or not), but potentially contribute to more efficient implementations or different perspective of ideas within them.
