\section{Benchmarking databases}\label{sec:background:dbbenchmarking}
Databases are pervasive to modern society and thus standards have arisen over
the last few decades to ensure that customers are able to pick their preferred
DBMS vendor.

At the top level, database benchmarks are classified into three categories: industry-standard, vendor and
customer-application \cite{PractitionersIntroduction}. 
These classifications are usually motivated by intention of the benchmark
instead of structure of the database management system; there is no shortage of
papers emphasising the importance of domain-specific benchmarks for applications
\cite{PractitionersIntroduction, BenchmarkHandbook} and depending on the
risk/performance tolerance of the application it may be necessary to consider
results from all three categories.

\paragraph{Vendor benchmark} A vendor database benchmark is used by
the database vendor during the production of the database management system.
This benchmark usually serves multiple purposes not just limited to the design
of the system. Of course, it is often used to highlight any performance
bottlenecks driving the design of the system internally but it usually doubles
up and acts as the knowledge base for the marketing of the system. Vendor
benchmarks are usually characterised by a more comprehensive suite of
tests to generate the insights capable of driving the direction of the
product \cite{PractitionersIntroduction}.

\paragraph{Industry-standard benchmark} An industry-standard benchmark is a set
benchmarking suite designed independently to any vendor or solution. It is
designed to allow for a fair comparison between different vendors and has been
shown to increase competition between vendors \cite{Wisconsin2}. As much of
this review will show, many industry-standard benchmarks have been developed
to give results for a wide range of applications of databases for much more
relevant and specialised results; and, although many benchmarks have similar
metrics \cite{SetQueryBenchmark, DebitCredit}, just selecting what metrics to
show consumers is not a trivial task \cite{DebitCredit}. When Gray's paper
\cite{BenchmarkHandbook} was written, it was noted that these benchmarks were
becoming so popular that vendors were also beginning to report their results
with their marketing.

\paragraph{Consumer-application benchmark} This type of benchmark refers to any
benchmarking that a customer would run, typically to choose between different
vendors for their application. This kind of testing can be critical for a
performance-sensitive application \cite{PractitionersIntroduction} and is often
done to test the performance of the database under a specific installation
profile \cite{DoingYourOwnBenchmark} or loads. This is a very specialised
requirements based benchmarking.

The review now turns to different domain-specific benchmarking of databases in
order to comment on and analyse any common structures found when designing
databases. This will help inform major design decisions of this project when
deciding what aspects to include in the benchmarking of the alternative join
syntax.

Despite the innumerable use cases for databases in the modern world, many
applications require and prioritise similar values when choosing a solution.
Furthermore, as most industrial-standard database benchmarks have been designed in
order to allow comparison for specific domain requirements, you will commonly
find database benchmarks that are designed with one of two different
applications in mind: \emph{transaction processing} or \emph{decision support}
\cite{PractitionersIntroduction}; it is worth mentioning that other application
types exist, however, such as \emph{document search} and \emph{direct marketing}
\cite{SetQueryBenchmark}. The application types of transaction processing and
decision support are so pervasive that they are commonly also used to partition
other application types, for instance OLTP (Online transaction processing) and
OLAP (Online analytical processing) are specialised for online processing but
share many comparative similarities as transaction processing and decision
support respectively \cite{OLTP-Oracle}.

This review will provide a brief overview of both
the main database application types, then focus more strongly on decision
support benchmarks as ad-hoc query processing is more heavily tested in these
benchmarks and thus more relevant to the project.

\subsection{Transaction processing}
Transaction processing is characterised by a large number of
update-intensive database services with a particular emphasis on integrity of
its requests \cite{PractitionersIntroduction}.
It is clear that this type of environment
demands an emphasis on throughput and integrity. For instance, a bank would
likely have an online transaction processing system in place and the
ability to deal with a large number of transactions in a short period of
time while maintaining the accuracy of the information is most likely paramount.
There are many well known benchmarks to test these types of systems,
including \emph{DebitCredit}, \emph{TPC-C} and \emph{TPC-E} \cite{DebitCredit,
TPC-OLTP}.
These benchmarks usually measure transactions per second or price per
transactions per second. Efforts have been made to try and describe what is
meant by ``price'' and usually it refers to the five-year capital cost (e.g.
cost of hardware, software and system maintenance over a five year period)
\cite{BenchmarkHandbook, DebitCredit}. The advantage of giving a metric
dependent on price is that price is likely to be a limiting of important factor
to most (if not all) organisations when shopping for database solutions and so
it helps encompass all information about the system into a single comparable
number \cite{SetQueryBenchmark}. It is worth noting that many benchmarks which
include this metric also give more information for the user \cite{TPC-H} or
allow the user to weight the comparative importance of features in this metric
\cite{SetQueryBenchmark} but it is expected that much emphasis will come down to
this defining number \cite{SetQueryBenchmark}.

We note that the above definition of a transaction processing system not only
highlights its intense performance requirements but also notes the importance of
integrity. The bank example highlights intuitively what we mean by integrity, a
shop may not want the transaction to go through zero times if it is reported
that it has, in fact, been processed, just as I would not want the transaction
to go through twice. So what exactly is a transaction? And how can we specify
its integrity in a rigorous way?

A transaction simply is an input to the database that must be considered as one
unit of work \cite{ComputerScienceDictionary} and completed independently to any
concurrent actions occurring \cite{ACID}. The integrity of this behaviour can be described 
by a set of properties typically remembered as the acronym \emph{ACID} and
explained below \cite{ACID, PractitionersIntroduction}.

\paragraph{Atomicity} An ``all or nothing'' property of a transaction. All
individual operations must take place or any partially-completed sequence of
operations must have no lasting effects on the data. The user should be informed
of which outcome has occurred.

\paragraph{Consistency} Given the database was in a consistent/valid state
before the transaction processed, it must remain in a consistent/valid state
after any changes are committed; that is to say the changes cause not integrity
constraints to be violated \cite{IntroToDatabaseSystems}. 
Some texts favour the term ``correctness'' \cite{IntroToDatabaseSystems} but this
discussion is outside of the scope of the project. 
% correctness vs consistency: correctness may be preferred as it is stronger, 
% and says that the database only contains true statements about the world

\paragraph{Isolation} Information related to the execution of one transaction
should be hidden from other transactions running concurrently. This is
equivalent to being able to find an order such that the results of the execution
of all transactions would be the same if they were executed sequentially in that
order.

\paragraph{Durability} Once a transaction has been successfully completed, its
effects must persist and future malfunctions.

\subsection{Decision support}
Decision support is characterised by the need to execute complicated queries on
a database with fewer changes \cite{IntroToDatabaseSystems,
PractitionersIntroduction}. The business value from this sort of application are
vast, especially in automatic report generation to produce insights and direct
company strategy \cite{SetQueryBenchmark, IntroToDatabaseSystems}. Despite the
clear business needs, the measurements usually taken by decision support
benchmarks are not as standard as those by transaction processing benchmarks and
metrics usually vary according to benchmark 
\cite{PractitionersIntroduction}; some common metrics collected are a measure of
elapsed time \cite{Wisconsin, TPC-H, PractitionersIntroduction}, performance of
the underlying system (CPU and I/O utilisation) \cite{PractitionersIntroduction}
and the throughput or price per throughput \cite{TPC-H, SetQueryBenchmark,
PractitionersIntroduction} (as in transaction support systems). 

Although less pervasive in our every day life to transaction support systems,
there are a large number of benchmarks for decision support systems. To name a
few there are the \emph{Wisconsin benchmark} \cite{Wisconsin}, \emph{Bull Single-User Decision
Support benchmark (SUDS)} \cite{PractitionersIntroduction}, \emph{Set Query
benchmark} \cite{SetQueryBenchmark} and \emph{TPC-H} \cite{TPC-H}. 

\subsection{Database benchmarking case studies}
In the section a variety of different benchmarks will be briefly presented and
compared. The reader should note that the list of benchmarks presented is
nowhere near exhaustive and are chosen specifically sample. Although the effort
for representation has been made, the benchmarks also were not chosen to be
representative of the large academic field; it must be noted that benchmarking
databases usually tests a lot more than is relevant to this project and so none
should be completely relevant. The decisions on which benchmarks to include in
this section were made on the following criteria:
\subparagraph{Historical importance}\label{enum:historicalImportance} I believe that the largest benefit of this
literature review to the project is identifying the steps that benchmarks have
taken to approach the empirically based and approximated rigorous approach they
are today in order to inform design decisions throughout the project. Therefore,
rather than just basing the decision on recency, I believe it more beneficial
considering papers of more historical importance in order to present an
intuition on key values within benchmarking databases instead of following a
clear and inexpressive specification that could lead to designs that obfuscate
the (specific) results the project deems important. 
\subparagraph{Relevance} Relevance to the project at hand is extremely
important. There are many defining features of the projects such as a query
based, join heavy theory and therefore benchmarks with specific abilities to
answer questions favouring these features were chosen (clearly leaning towards
decision support benchmarks).
\subparagraph{Recency} The recency of the benchmark was taken into consideration
when comparing to others, however was not a deciding factor. At times, if there
were multiple similar benchmarks without historical importance (especially if
it has been made clear that some benchmarks have been deprecated) the most
recent benchmark was chosen. This was to try and consider the most up to date
practices when exploring this field.

\subsubsection{Wisconsin}
In Bitton's paper, ``Benchmarking Database Systems - A Systematic Approach''
\cite{Wisconsin} she presents the \emph{Wisconsin benchmark} alongside a
comparison of a few major database systems at the time. She claimed that her
custom database and queries represented a ``first attempt in developing a
scientific methodology for performance evaluation of a database management
system'' and claimed that previous evaluation studies of database machines were
``based on simplified analytical models''.

This literature review is focused only on the benchmarking methodology and
largely ignores any results obtained by Bitton for the now clearly outdated
database machines. From my perspective, her main contribution to the field was
the idea to construct a synthetic database for the database which has
subsequently been used in almost all benchmarks since, with the Set Query
Benchmark specifically pointing out this contribution in its acknowledgements
\cite{SetQueryBenchmark}. 

Bitton reflects on the issues of using empirical
databases for benchmarking observing that not only would you require a large
sample size before they could be considered randomly distributed; furthermore
for a benchmark empirical databases are difficult to scale and make it difficult
to specify the selectivity of the operations being benchmarked. Selectivity has
been used loosely here to refer to any operation that acts on a subset of
the tuples; as Bitton explains, it is difficult to construct a selection query that
selects a tenth of the database (i.e. a selection query with an \emph{selectivity
factor} of 10\%) or only keeps 1000 tuples in an empirical database. Bitton
notes that modelling selectivity factors on queries with joins is especially
difficult (a key part of our problem domain). 

Another benefit of using a synthetic database is, given that it is designed
well, it should be more readable. The Wisconsin benchmark boasts relations and
attributes with names that immediately give you an idea of what you would find
and the distribution of its values. In Jim Gray's later work
\cite{BenchmarkHandbook}, he specifically calls out simplicity as one of four
key properties of a domain-specific benchmark - defining it as a database's
understandability and commenting that without it the database would lack
credibility.

\paragraph{The relations} This section contains a description of the relations in
the database system and a brief description of their use. The database consists
of 4 relations, namely ``\relation{thoustup}'', ``\relation{twothoustup}'',
``\relation{fivethoustup}'', and ``\relation{tenthoustup}'' with 1000, 2000, 5000
and 10000 tuples respectively; where it is
clear that the relations are named after it's cardinality (the number of tuples
it contains \cite{PractitionersIntroduction}). It is worth noting that multiple
relations are generated for use in 
queries with multiple operands (such as a join), producing in practice
``tenthoustupA'' and ``tenthoustupB'' for instance. The advantage to this explicit
naming is as above, any implementer knows exactly what relation they are dealing
with making the benchmark a much more readable endeavour. Furthermore, as you will
see later, it is paramount to calculating selectivity factors.

\paragraph{The tuples} Each relation has 16 attributes, designed such that a
tuple is exactly 182 bytes long in order to make the size of all four relations
approximately 4 MB, according to Bitton. These attributes can be found in
\fref{tab:WisconsinAttributes} and are comprised of 2 key-like unique integer
attributes, 12 cyclic integer attributes with varying domains and 3 string attributes.

\begin{table}[t]
    \begin{tabular}{r c c p{0.2\linewidth}}
        \toprule
        \textbf{Attribute} & \textbf{Type} & \textbf{Domain} &
        \textbf{Properties}\\
        \midrule

        \relationAttribute{unique1} & integer & $\{0, \ldots, K - 1\}$ & random, unique, key \\
        \relationAttribute{unique2} & integer & $\{0, \ldots, K - 1\}$ & random, unique, key,
        primary \\
        \midrule

        \relationAttribute{two} & integer & $\{0, 1\}$ & cyclic \\
        \relationAttribute{four} & integer & $\{0, \ldots, 3\}$ & cyclic \\
        \relationAttribute{ten} & integer & $\{0, \ldots, 9\}$ & cyclic \\
        \relationAttribute{twenty} & integer & $\{0, \ldots, 19\}$ & cyclic \\
        \relationAttribute{hundred} & integer & $\{0, \ldots, 99\}$ & cyclic \\
        \relationAttribute{thousand} & integer & $\{0, \ldots, 999\}$ & cyclic \\
        \relationAttribute{twothous} & integer & $\{0, \ldots, 1999\}$ & cyclic \\
        \relationAttribute{fivethous} & integer & $\{0, \ldots, 4999\}$ & cyclic \\
        \relationAttribute{tenthous} & integer & $\{0, \ldots, 9999\}$ & cyclic \\
        \relationAttribute{odd100} & integer & $\left\{n\ |\ n \leftarrow \{0, \ldots, 99\},\
        \mathrm{odd}\  n\right\}$ & random \\
        \relationAttribute{even100} & integer & $\left\{n\ |\ n \leftarrow \{0, \ldots, 99\},\
        \mathrm{even}\  n\right\}$ & random \\
        \midrule

        \relationAttribute{stringu1} & \CharStringWithLength{52}
                             & \WisconsinUStringDef
                             & unique, derived from \relationAttribute{unique1}, key\\

        \relationAttribute{stringu2} & $\mathrm{char}(52)$  
                             & \WisconsinUStringDef
                             & unique, derived from \relationAttribute{unique2}, key, primary\\

        \relationAttribute{string4} & $\mathrm{char}(52)$  
                            & $\WisconsinLabelledStringStructure{\$}{\$}{\$},\ \$ \in
                            \{A, H, O, V\}$
                            & cyclic \\
        \bottomrule
    \end{tabular}

    \caption{The attributes of a relation with cardinality $K$ in the Wisconsin
    database for benchmarking \cite{PractitionersIntroduction, Wisconsin}}
    \label{tab:WisconsinAttributes}
\end{table}

\subparagraph{Unique integer attributes} The unique attributes, names \relationAttribute{unique1} and
\relationAttribute{unique2} are integer fields that range from $0$ to $K - 1$ where $K$
is the cardinality (number of tuples) of the relation. \emph{Both} fields are
randomly generated (as is emphasised in Bitton's reflective paper
\cite{Wisconsin2}) but it is noted that \relationAttribute{unique2} is usually used to
sort the relation. As the name suggests, both fields are unique. 

\subparagraph{Cyclic integer attributes} These names of these attributes are
named after the number of values they can take and any properties; for instance
``\relationAttribute{four}'' can take 4 values from 0 to 3, and ``\relationAttribute{even100}''
can take any even number between 0 and 99 (the first 100 numbers counting from
0). The possible values for each of these attributes are cycled through while
generating the table. These attributes are key to the
advantages of using a synthetic database as, given the relation cardinality, you
can use it to specify any number of selectivity factors. For instance a
selection operation on \relationAttribute{two} relation would result in a selectivity
factor of 50\%. 

\subparagraph{String attributes} The database has three string attributes, all
of which contain 52 characters. Two of these attributes, \relationAttribute{stringu1}
and \relationAttribute{stringu2} are derivative on other randomly generated attributes
in the relation, namely \relationAttribute{unique1} and \relationAttribute{unique2}
respectively. Whereas the final string attributes \relationAttribute{string4} cycles
through four variations of a string. It is noted that all strings have names
that help the reader understand their origins, with the first two containing
``u1'' or ``u2'' denoting what component the value would be derived from and the
``4'' in \relationAttribute{string4} denoting the cardinality of its domain. Each string follows
the template outlined in \fref{fig:WisconsinStringTemplate}; each \$ is to be
replaced with a letter between A and V inclusive (the \emph{significant
letters}) and x denotes the character literal
`x'. As each of the \$ can be replaced independently with 22 different
letters, the template can produce $22^3 = 10,648$ unique strings. The design
intention for this string was twofold, firstly it is clear how the template
could easily be expanded to allow for more significant letters but less
obviously, the spacings of the significant letters were meant to help change the
CPU load by changing the CPU time needed to compare two strings.

\begin{figure}[h]
    \centering
    \WisconsinUStringDef 
    \caption{The general structure of a string attribute in the Wisconsin
        database. Each \$ is to be substituted by a letter between A and V, and x
    denotes the character literal `x'.}
    \label{fig:WisconsinStringTemplate}
\end{figure}

Both \relationAttribute{stringu1} and \relationAttribute{stringu2} are derived from
\relationAttribute{unique1} and \relationAttribute{unique2} in the same way. The string
corresponding to 0 is \WisconsinStringStructure{A}{A}{A} and to find the
subsequent string, the leftmost character is `increased' and once it reaches `V',
the next character increased (in the same way) and the current character is
reset to `A`. This can be seen counting in base 22 with the least significant
digit on the left instead of the right. An example can be found in
\fref{fig:WisconsinUniqueStringExample}.

\begin{table}[h]
    \begin{tabular}{c c}
        \toprule
        \relationAttribute{unique1} & \relationAttribute{stringu1} \\
        \midrule
        
        0 & \WisconsinStringStructure{A}{A}{A} \\
        1 & \WisconsinStringStructure{B}{A}{A} \\
        2 & \WisconsinStringStructure{C}{A}{A} \\

        \vdots & \vdots \\

        21 & \WisconsinStringStructure{V}{A}{A} \\
        22 & \WisconsinStringStructure{A}{B}{A} \\
        23 & \WisconsinStringStructure{B}{B}{A} \\

        \vdots & \vdots \\

        12,166 & \WisconsinStringStructure{V}{V}{A} \\
        12,167 & \WisconsinStringStructure{A}{A}{B} \\
        \vdots & \vdots \\
        \bottomrule
        
    \end{tabular}
    \caption{An example of the process to derive \relationAttribute{stringu1} values
    from \relationAttribute{unique1} values.}
    \label{fig:WisconsinUniqueStringExample}
\end{table}


The components of \relationAttribute{string4} simply cycle through the 4 values
\WisconsinRepeatedStringStructure{A},
\WisconsinRepeatedStringStructure{H},
\WisconsinRepeatedStringStructure{O},
\WisconsinRepeatedStringStructure{V}.
If any ordering is required it is done on \relationAttribute{unique2} (or equivalently
\relationAttribute{stringu2}).

\paragraph{Improvements to the Wisconsin Database} As one of the first
standard ways of testing database systems the Wisconsin database has been
criticised throughout the years and been subject to multiple retrospectives by
the original authors \cite{Wisconsin2}. Among other comments there were emphases
on criticism revolving around the single user limitation of the benchmark, its
lack of scalability and string design. Not addressing the single user concerns
in this benchmark improvements were suggested for the latter two issues. In
order to address the scalability concerns and string design Bitton introduces a
change to a few of their attributes in \cite{Wisconsin2} as well as rework which
can be seen in \fref{tab:WisconsinImprovedAttributes}. A brief non-exhaustive
description of the changes follow.

\subparagraph{Unique attributes} The unique attributes were changed so that only
\relationAttribute{unique1} is randomly generated and
\relationAttribute{unique2} is sequential.

\subparagraph{Cyclic integer attributes} The cyclic integer attributes such as
\relationAttribute{two} and \relationAttribute{hundred} were mostly replaced
with percentages. The cardinalities stay the same, but the method of generation
also differs: instead of cycling through values, they are now randomly generated
by using the appropriate modulo operation on \relationAttribute{unique1}. The
reason for the change is that you can exactly tell the selectivity of a query;
for example, as described by Bitton, you know that a selection on
\relationAttribute{twentyPercent}, say $\relationAttribute{twentyPercent} = 2$,
will give 20\% of the table no matter the cardinality.

\subparagraph{String attributes} The string attributes remained the same size
but their template and methods of generation changed. The justification for the
fixed-length strings was so that each disk page could hold the same number of
tuples. An an overview of the template for \relationAttribute{stringu1} and
\relationAttribute{stringu2} can be found in \fref{fig:NewWisconsinUniqueString}
and for \relationAttribute{string4} in \fref{fig:NewWisconsinCyclicString}. The
generation are not described but it suffices to know that
\relationAttribute{stringu1} and \relationAttribute{stringu2} are generated
based on \relationAttribute{unique1} and \relationAttribute{unique2}
respectively, whereas \relationAttribute{string4} continues to be cyclic.

\begin{figure}[h]
    \centering
    \NewWisconsinUString

    \caption{Template for the unique strings in the updated Wisconsin database.}
    \label{fig:NewWisconsinUniqueString}
\end{figure}

\begin{figure}[h]
    \centering
    \NewWisconsinCyclicStringDef{}

    \caption{Template for \relationAttribute{string4} in the updated Wisconsin database.}
    \label{fig:NewWisconsinCyclicString}
\end{figure}


\begin{table}[t]

    \begin{tabular}{l c l}
        \toprule
        Attribute & Values & Properties \\
        \midrule

        \relationAttribute{unique1} & $\{0, \ldots, K - 1\}$ & random, unique \\
        \relationAttribute{unique2} & $\{0, \ldots, K - 1\}$ & sequential, unique \\
        \midrule

        \relationAttribute{two} & $\{0, 1\}$ & random,
        \NewWisconsinCyclicIntegerDef{2} \\
        \relationAttribute{four} & $\{0, \ldots, 3\}$ & random,
        \NewWisconsinCyclicIntegerDef{4} \\
        \relationAttribute{ten} & $\{0, \ldots, 9\}$ & random,
        \NewWisconsinCyclicIntegerDef{10} \\
        \relationAttribute{twenty} & $\{0, \ldots, 19\}$ & random,
        \NewWisconsinCyclicIntegerDef{20} \\
        \relationAttribute{onePercent} & $\{0, \ldots, 99\}$ & random,
        \NewWisconsinCyclicIntegerDef{100} \\
        \relationAttribute{tenPercent} & $\{0, \ldots, 9\}$ & random,
        \NewWisconsinCyclicIntegerDef{10} \\
        \relationAttribute{twentyPercent} & $\{0, \ldots, 4\}$ & random,
        \NewWisconsinCyclicIntegerDef{5} \\
        \relationAttribute{fiftyPercent} & $\{0, 1\}$ & random,
        \NewWisconsinCyclicIntegerDef{2} \\
        \relationAttribute{unique3} & $\{0, \ldots, K - 1\}$ & random,
        \relationAttribute{unique1} \\
        \relationAttribute{evenOnePercent} & $\{0, 2, \ldots, 198\}$ & random,
        $\left(\relationAttribute{onePercent} \cdot 2\right)$ \\
        \relationAttribute{oddOnePercent} & $\{1, 3, \ldots, 199\}$ & random,
        $\left(\relationAttribute{onePercent} \cdot 2\right) + 1$ \\
        \midrule

        \relationAttribute{stringu1} & \NewWisconsinUString{} & random, based on
        \relationAttribute{unique1} \\
        \relationAttribute{stringu2} & \NewWisconsinUString{} & random, based on
        \relationAttribute{unique2} \\
        \relationAttribute{string4} & \NewWisconsinCyclicStringDef{} & cyclic \\
        \bottomrule
    \end{tabular}


    \caption{The improved attributes of the Wisconsin database as seen in
    \cite{Wisconsin2}.}
    \label{tab:WisconsinImprovedAttributes}
\end{table}

\paragraph{Queries} Along with a synthetic database, the Wisconsin benchmark
also provides a set of queries to benchmark different relational operators.
Bitton's queries attempt to measure the performance of the following (as stated
in \cite{Wisconsin})
\begin{enumerate}
    \item Selection queries with varying selectivity factors
    \item Projections with different proportions of duplicated attributes (whose
        performance impact comes from the collapsing of duplicates as described
        in \fref{sec:projections})
    \item Single and multiple joins
    \item Simple aggregates and aggregate functions
    \item Updates
\end{enumerate}
Many queries contain version for using both primary and secondary indices.

\subsubsection{Set Query Benchmark}
The Set Query Benchmark as described by O'Neil \cite{SetQueryBenchmark} is a
benchmark that aims to use a more realistic set of operations for decision
support benchmarks than TPC and DebitCredit, namely the \emph{set queries}. Set queries
are queries which potentially need information from a large set of tuples.

\paragraph{The BENCH relation} The table used in the set benchmark, named
\relation{BENCH}, is synthetically generated with 21 columns. Similarly to the
Wisconsin benchmark there are a mixture of integer and string attributes; Namely
13 indexed attributes and eight character columns. The \relation{BENCH} table
always has a multiple of 1 million rows, and for the information below we assume
the default of 1 million rows. Each row has 200 bytes each.

\paragraph{Indexed columns} The thirteen indexed columns are all integers,
twelve of which are ordered. The attributes are as follows
\relationAttribute{KSEQ}, \relationAttribute{K500K},
\relationAttribute{K250K}, \relationAttribute{K100K},
\relationAttribute{K40K}, \relationAttribute{K10K},
\relationAttribute{K1K}, \relationAttribute{K100},
\relationAttribute{K25}, \relationAttribute{K10}, \relationAttribute{K5},
\relationAttribute{K4}, \relationAttribute{K2}. All columns except
\relationAttribute{KSEQ} are unordered (randomly generated) whereas
\relationAttribute{KSEQ} enumerates the columns in order from 1. For the
unordered columns, such as \relationAttribute{K100} the number denotes its
cardinality (where if a number ends with a ``K'' it denotes a multiple of 1000,
such as \relationAttribute{K40K} having a cardinality of $40,000$). If a column
has a cardinality of $N$, then components can take any value between $1$ and
$N$ inclusive. In order to allow the scaling of tables the cardinality of the
two largest ordered columns, \relationAttribute{K500K} and
\relationAttribute{K250K}, increase proportionally to the number of rows; as
\relationAttribute{KSEQ} enumerates the rows it is clear that its
cardinality also increases as the number of rows increases.

\paragraph{Character columns} The character columns in the Set Query Benchmark
serve a very different purpose to the strings in the Wisconsin database and are
there simply to pad out a row to 200 bytes. The columns are named simply
\relationAttribute{S1}, \relationAttribute{S2}, \ldots, \relationAttribute{S8}.
The relation
\relationAttribute{S1} is an 8 character string, whereas \relationAttribute{S2} through to
\relationAttribute{S8} are 20 character strings. The strings are not used for retrievals
and so O'Neil says they can all be set to the same value; interestingly, O'Neil
writes that the reason they are not used in retrievals is to mirror the fact
that
retrievals in a large database system with scarce writes on columns without
indexing is generally not advised.

\subsubsection{TPC-H}
The TPC-H benchmark is one of the more recent benchmarks for decision support
systems designed by the Transaction Processing Performance Council (TPC). The
benchmark contains a bigger emphasis on updates compared to Wisconsin and Set
Query Benchmark (specifically concurrent updates) while still focusing on
ad-hoc business related queries. This benchmark is meant to represent a more
modern and authoritative benchmarking system in this literature review but is
not to be discussed in depth as it is beyond the scope of this project.

\paragraph{Overall Structure} The TPC-H benchmark takes a different, perhaps more
realistic, approach to their database model. The benchmark consists of eight
``separate and individual tables'' which have varying relationships to each
other. The database has some constraints on size, such as a minimum on data from
10,000 suppliers which leads to a database of almost ten million rows
(approximately a raw storage capacity of 1 GB). This is default size not
comparable to either the Wisconsin benchmark and the set query benchmark; it is
worth arguing that both can be scaled up, but so too can the TPC-H benchmark be
scaled to much larger datasets.

\paragraph{The relations} The table definition clearly has parallels to the business world with
tables not only being assigned varying structures and cardinalities, but
real-life applications too. We see, for instance, the \relation{SUPPLIER}
relation, with attributes such as \relationAttribute{SUPPKEY},
\relationAttribute{NAME} and \relationAttribute{ACCTBAL}. The cardinality of
\relation{SUPPLIER} is also specified as $\mathrm{SF} \cdot 10,000$ where
$\mathrm{SF}$ is the scale factor for the chosen database size. This contrasts
the \relation{ORDERS} relation with attributes such as
\relationAttribute{TOTALPRICE} and
a cardinality of $\mathrm{SF} \cdot 1,500,000$.

\subsection{Best practices}\label{sec:background:benchmarkbestpractices}
It is clear that after the idea worked so well in the Wisconsin benchmark, a
synthetic database is vital to a well done benchmark of DBMS. Despite the wide
array of different takes to benchmarking, both due to progressions in the field
and difference in requirements there is a large overlap in what values each
designer chose. This subsection is designed to condense the lessons in the
previous section.

\paragraph{Synthetic databases} Synthetic databases are key to give benchmarks
the customisation, standards, readability and structure necessary to provide any
sort of useful result. We see many approaches to the synthetic database
structure, the multiple tables for relation sizes seen in
Set Query Benchmark \cite{SetQueryBenchmark}, the single table whose size is changed via selection
queries in Wisconsin \cite{Wisconsin}, and the realistic approach set up in
\cite{TPC-H}. The size of each tuple can be designed so that a relation's size
is known. And perhaps most importantly, selectivity factors and CPU usage can be
changed by definition or combination of attributes.

\paragraph{Scalability} A key criticism for founding DBMS benchmark systems is a
lack of scalability, as can be seen in the update to the Wisconsin benchmark
\cite{Wisconsin2}. Technology is advancing often as well as vastly different
needs in different domains, and so a `one size fits all' approach is not viable
for such benchmarks. A general strategy employed to overcome any of these
limitations is to partition a percentage of the database or have the largest $n$
attributes scale to the size of the relation. It is also worth noting that
scalability to parallel execution is also desirable \cite{BenchmarkHandbook}.

\paragraph{Readability} The Wisconsin benchmark \cite{Wisconsin} was one of the
first to emphasise the importance of making the relation easy for the reader to
understand. A good benchmark should be able to be read and understood easily
with any new, perhaps more relevant, queries added easily. Similarly, Jim Gray
talks about the value of simplicity for the user, linking its understandability
to its credibility \cite{BenchmarkHandbook}.

\paragraph{Relevance} Jim Gray believes that relevance is important when
designing a system and specifies metrics to be reported in order to fulfil such
a requirement, namely ``peak performance and price/performance of systems when
performing typical operations within that problem domain''. For this project
these specific recommendations should be taken with a grain of salt as they are
not completely indicative of the desired outcome however the idea of relevance
is key. Furthermore, some benchmarks such as TPC-H \cite{TPC-H} have also taken
relevance in the structure of their database which may be more attractive for
those working in the same or adjacent domains.

\paragraph{Portability} Jim Gray also emphasises the need to implement the
benchmark on different systems and architectures \cite{BenchmarkHandbook}. You
can see varying approaches to this, in early Wisconsin benchmark and the time it
was created, the architectures were closely linked to the DBMS systems
\cite{Wisconsin} and so there was a more closed minded scope. However, reading
the literature you can see the portability of benchmarks evolve as the
technology does, notably defining queries in the very portable language SQL
\cite{Wisconsin2, SetQueryBenchmark}.
