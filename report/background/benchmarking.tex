\section{Benchmarking databases}
Databases are pervasive to modern society and thus standards have arisen over
the last few decades to ensure that customers are able to pick their preferred
DBMS vendor.

At the top level, database benchmarks are usually split into three different
categories: industry-standard, vendor and
customer-application \cite{PractitionersIntroduction}. Each type of benchmark
usually is driven by a different goal and thus are all important to the creation
of a DBMS. Briefly, a vendor database benchmark is usually used during the
development of a DBMS by the vendor themselves to aid
the general development of the solution and sales; it is usually characterised
by a more comprehensive suite of tests driven by insights into the internals. An
industry-standard is usually characterised by independent testing of the
database for comparisons across vendors. Finally, a customer-application test is
a test that an organisation should design and run before choosing a specific
vendor, choosing to test specific requirements and emphases related to their
business domain.


The type of benchmark to run also largely depends on the use-case for the
database. Typically we can refer to the use cases as \emph{transaction
processing} or \emph{decision support} \cite{PractitionersIntroduction}.

\subsection{Transaction processing}
Transaction processing is characterised by a large number of
update-intensive requests
\cite{PractitionersIntroduction}. It is clear that this type of environment
demands an emphasis on throughput and integrity. For instance, a bank would
likely have an \emph{online transaction processing system} or OLTP system in
place and it clear that the ability to deal with a large number of transactions
in a short period of time and for the accuracy if the information given is
paramount. There are many well known benchmarks to test these types of systems,
including \emph{DebitCredit}, \emph{TPC-C} and \emph{TPC-E} \cite{TPC-OLTP}.
These benchmarks usually measure transactions per second.

\subsubsection{Transaction integrity}
We briefly note what we mean by the integrity of a transaction. A transaction
simply is an input to the database that must be considered as one unit of work
\cite{ComputerScienceDictionary} and completed independently to any concurrent
actions occurring. This behaviour can be described a set of
properties typically remembered as the acronym \emph{ACID}
\cite{ComputerScienceDictionary, PractitionersIntroduction}.
\paragraph{atomicity}
\paragraph{consistency}
\paragraph{isolation}
\paragraph{durability}

\subsection{Decision processing}
\subsubsection{Wisconsin benchmark}

\subsubsection{Single-User Decision Support} % SUDS

\subsubsection{The Set Query Benchmark}

\subsection{Best practices benchmarking}
