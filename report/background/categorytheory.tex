\section{Category theory}
Category theory will be our main tool in describing the mathematical structure of different elements of our database systems, and relations between them. More generally, category theory can be seen as a way of taking the abstractions that algebra was built on to a higher level, an ``abstraction of abstractions''. You can find that just as easily as categories can help us in our domain, categories have a rich language and can describe many structures in mathematics ranging from groups and rings to matrices. 
\subsection{Categories}
\theoremstyle{definition}\newtheorem*{categorydef}{Category}
We will first take the purely mathematical introduction to category theory. We see that most structures in mathematics have similar key features: a collection of elements (typically with some rules governing them depending on the definition of the structure) and morphisms or transformations between them preserving the structure of elements. Notice, groups and group homomorphisms, rings and ring homomorphisms, topological spaces and continuous maps. This will be our inspiration while defining categories, ultimately the abstraction of these structures. \todo{introduce the non maths way of looking at it as well from the categories, types and structures book}
\begin{categorydef}
  A \emph{category} \cat{C} is a set\footnote{In more rigorous definitions one must be careful of defining the collections of objects as a set lest Russell's paradox comes into play}\todo{Make sure that this is correct.} of \emph{objects} \objs{C}, such as \obj{a}, \obj{b}, \obj{c}, and \emph{morphisms} (or \emph{arrows}) \morphs{C} between them, such as \morph{f}, \morph{g}. We require that:
  \begin{itemize}
    \item There are two operations; \emph{domain} which associates with every arrow \morph{f} an object $\obj{a} = \domain{f}$ and \emph{codomain} which associates with every arrow \morph{f} an object $\obj{b} = \codomain{f}$. We can now express this information as \explicitmorph{f}{a}{b}.\footnote{Though we emphasise the distinction between a function and a morphism.}
    \item There is a composition rule between morphisms such that given \explicitmorph{f}{a}{b} and \explicitmorph{g}{b}{c}, there is another arrow \explicitmorph{\morph{g} \circ \morph{f}}{a}{c} in \morphs{C}.
    \item Composition of arrows is associative. That is, for an additional object \obj{d} and arrow \explicitmorph{h}{c}{d} the resulting morphisms $\morph{h} \circ \left(\morph{g} \circ \morph{f}\right)$ and $\left(\morph{h} \circ \morph{g}\right) \circ \morph{f}$ coincide in \morphs{C}.
    \item Every object \obj{a} is assigned an arrow $\id{a}: \obj{a} \to \obj{a}$ in \morphs{C}, called the \emph{identity morphism}.
    \item Composition with the identity morphism is the identity on morphisms. Explicitly, given the arrow \explicitmorph{f}{a}{b}, we have $\morph{f} \circ \id{a} = \id{b} \circ \morph{f} = \morph{f}$.
  \end{itemize}
\end{categorydef}
\todo{introduce hom-set}

\subsection{Functors}
\theoremstyle{definition}\newtheorem*{covfunctordef}{Functor}

Similar to the arrows we have within categories, we can transform categories to other categories. This is the notion of a functor.
\begin{covfunctordef}
  Given two categories \cat{C} and \cat{D}, a (covariant) functor $\functor{F}: \cat{C} \to \cat{D}$ is two related functions, an \emph{object function} and an \emph{arrow function} (both written as \functor{F}). The object function $\functor{F}: \objs{C} \to \objs{D}$ maps an object $a$ in \cat{C} to $\functorobj{F}{a}$ in \cat{D}. Given an arrow $\morph{f}: \obj{a} \to \obj{b}$ in \cat{C}, the arrow function maps \morph{f} to $\functormorph{F}{f}: \functorobj{F}{a} \to \functorobj{F}{b}$. The functor must obey the following properties: \todo{Should I say the arrow is in the cat or the morphs of the cat}
  \begin{itemize}
    \item It must preserve the identity morphism, i.e. $\functormorph{F}{\id{a}} = \id{\functorobj{F}{a}}$.
    \item It must also preserve composition, i.e. $\functormorph{F}{\left(\morph{g} \circ \morph{f}\right)} = \functormorph{F}{\morph{g}} \circ \functormorph{F}{\morph{f}}$.
  \end{itemize}
\end{covfunctordef}

Intuitively, it should be obvious that categories and functors themselves form a category, \commoncatname{Cat}.\\

\subsection{Natural transformations}
\theoremstyle{definition}\newtheorem*{nattransdef}{Natural Transformation}
We now can explore an intuitive way of relating two functors, called an \emph{natural transformation}. A natural transformation can be seen as a projection of one functor space to another and is in essence a family of arrows that describes this translation.
\begin{nattransdef}
Given two functors $\functor{F}, \functor{G} : \cat{C} \to \cat{D}$ a
\emph{natural transformation} $\explicitnattrans{\tau}{F}{G}$ associates to each element $\obj{a} \in \objs{C}$ an arrow $\nattransapply{\tau}{a} \in \morphs{D}$ such that $\nattransapply{\tau}{a}: \functorobj{F}{a} \to \functorobj{G}{a}$. Additionally, for $\obj{b} \in \cat{C}$ and \explicitmorph{f}{a}{b} a morphism in \cat{C}, we also require that $\nattransapply{\tau}{b} \circ \functormorph{F}{f} = \functormorph{G}{f} \circ \nattransapply{\tau}{a}$.

\end{nattransdef}

\subsubsection{Compositions}
As you might expect from the denotation of a natural transformation with an
arrow, there are composition laws for them. They are slightly more complicated
than most compositions, however, and have two different kinds.
\theoremstyle{definition}\newtheorem*{verticalcompositiondef}{Vertical Composition}
\theoremstyle{definition}\newtheorem*{horizontalcompositiondef}{Horizontal Composition}

\cite{RelationalAlgebraByWayOfAdjunctions}
\begin{verticalcompositiondef}
    Given two natural transformations
    \explicitnattrans{\tau}{E}{F} and
    \explicitnattrans{\nu}{F}{G} where $\functor{E, F, G}: \cat{C} \to
    \cat{D}$ are functors. The vertical composition
    $\explicitnattrans{\verticalcomposition{\tau}{\nu}}{E}{G}$
    is defined by the composition of the underlying arrows for every
    object $\obj{A} \in \cat{C}$. Explicitly, the components
    $\nattransapply{\left(\verticalcomposition{\tau}{\nu}\right)}{A} =
    \morphcomp{\nattransapply{\tau}{A}}{\nattransapply{\nu}{A}}$
\label{sec:verticalcomposition}
\end{verticalcompositiondef}

\todo{Add horizontal composition definition}


\subsection{Adjunctions}
An adjunction expresses an intersection of arrows of two different functions. For instance Say you have the category \commoncatname{Set} of sets \cite{RelationalAlgebraByWayOfAdjunctions} and two functions $\morph{f}, \morph{g}$ with signatures \explicitmorph{f}{X}{A} and \explicitmorph{g}{X}{B} with $\obj{X}, \obj{A}, \obj{B} \in \commoncatname{Set}$.
in \commoncatname{Set} with a common domain $\domain{f} = \domain{g} = \obj{A} \in \commoncatname{Set}$, how might we interpret the application of both functions on one element. We could create a new function \explicitmorph{h}{X}{A \times B} which for all $a \in \obj{A}$ maps $a \mapsto (\morph{f}(a), \morph{g}(a))$. Note that both \obj{X} and $\obj{A} \times \obj{B}$ are sets (where $\times$ denotes the cartesian product of two sets, i.e. the set of all such pairings of elements of \obj{A} and \obj{B}). This is not a very natural way of writing this however. Consider now the product category $\commoncatname{Set}^2$ such that the elements are just pairs of sets and functions, pairs of functions $(\morph{i}, \morph{j}): (\domain{i}, \domain{j}) \to (\codomain{i}, \codomain{j})$. It is clear that \commoncatname{Set} and $\commoncatname{Set}^2$ are distinct, i.e. a pair of a set is not a set nor is a pair of set. However, interestingly $h$ can quite easily be expressed within $\commoncatname{Set}^2$ as the function $(f, g)$ that maps $(\obj{A}, \obj{A})$ to $(\obj{X}, \obj{Y})$ and links back to the situation in $\commoncatname{Set}$ become clear. We introduce the diagonal functor $\functor{\Delta}: \commoncatname{Set} \to \commoncatname{Set}^2$, s.t. $\functorobj{\Delta}{A} = (\obj{A}, \obj{A})$ and $\functormorph{\Delta}{f} = (\morph{f}, \morph{f})$. Furthermore, the cartesian product can also be seen view the lens of a functor $\functor{\times}: \commoncatname{Set}^2 \to \commoncatname{Set}$ that takes the pair of elements $(\obj{A}, \obj{B})$ and maps it to the set $\obj{A} \times \obj{B}$ and a pair of functions $(\morph{i}, \morph{j})$ to a function $k: \left(\domain{i} \times \domain{j}\right) \to \left(\codomain{i} \times \codomain{j}\right)$.
Considering the above problem again, we can see very clear links between the two functors. We can work our problem in \commoncatname{Set} in the domain of $\commoncatname{Set}^2$ by considering $\functorobj{\Delta}{A} = (\obj{A}, \obj{A})$. We can then easily apply $(\morph{f}, \morph{g})$. Similarly, given the problem in $\commoncatname{Set}^2$ we see that we are left with a pair of sets that can easily be considered a cartesian product $\obj{X} \times \obj{Y}$. It should be clear that there is a bijection between the arrows $\functorobj{\Delta}{A} \to (\obj{X}, \obj{Y})$ in $\commoncatname{Set}^2$ and the arrows $\obj{A} \to \functorobj{(\times)}{(X, Y)}$.

\theoremstyle{definition}\newtheorem*{adjunctiondef}{Adjuction}
Motivated by this example, we give a formal definition of an adjunction.
\begin{adjunctiondef}
    Given two functors $\functor{L}: \cat{D} \to \cat{C}$ and $\functor{R}:
    \cat{C} \to \cat{D}$, we define an adjunction $\adunction{L}{R}$ \todo{has a
    domain and codomain} such that there is a natural isomorphism between the
    hom-sets as follows:\cite{RelationalAlgebraByWayOfAdjunctions}
    \[
        \lfloor - \rfloor: \homset{C}{\functorobj{L}{A}}{B} \cong
        \homset{D}{A}{\functorobj{R}{B}} :\lceil - \rceil
    \]
    We call \functor{L} the \emph{left adjoint} and \functor{R} the \emph{right
    adjoint}.
    The natural transformations between the arrows of the categories are defined
    by the \emph{unit}: \explicitnattrans{\eta}{\mathrm{Id}}{R \circ L}, $\nattransapply{\eta}{A} =
    \lfloor \id{\functorobj{L}{A}} \rfloor$ and, symmetrically, the
    \emph{counit}:
    $\explicitnattrans{\epsilon}{L \circ R}{\mathrm{Id}}$ such that
    $\nattransapply{\epsilon}{B} = \lceil \id{\functorobj{R}{A}} \rceil$. We
    require that the unit and counit obey the `triangle identities':
\[
            \verticalcomposition{\nattransapply{\eta}{\functor{R}}}{\functorobj{R}{\nattrans{\epsilon}}}
            = \id{}
\] \todo{fix notation}
    and
\[
    \verticalcomposition{\nattransapply{\eta}{\functor{L}}}{\functorobj{L}{\eta}}
    = \id{}
\]
\end{adjunctiondef}

We see that the unit $\eta$ is a mapping between elements in \cat{D}, noting
especially the difference between the functor \functor{\mathrm{Id}} and the
morphism \id{\functorobj{L}{A}}.
In trying to understand the triangle inequalities it is worth remembering the
definition of vertical composition in \fref{sec:verticalcomposition}, namely
that
$\nattransapply{\left(\verticalcomposition{\eta}{\epsilon})\right)}{\obj{A}} =
\morphcomp{\nattransapply{\eta}{\obj{A}}}{\nattransapply{\epsilon}{\obj{A}}}$
where the right hand side is simple the composition of arrows.
\todo{Description very similar to the paper so make sure it is not plagiaris}

\todo{Describe why adjunctions are such a key character in this paper}
\todo{Add more information between here}
\subsection{Graded Monads}\label{sec:gradedmonads}
