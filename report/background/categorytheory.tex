\section{Category theory}
Category theory will be our main tool in describing the mathematical structure of different elements of our database systems, and relations between them. More generally, category theory can be seen as a way of taking the abstractions that algebra was built on to a higher level, an ``abstraction of abstractions''. You can find that just as easily as categories can help us in our domain, categories have a rich language and can describe many structures in mathematics ranging from groups and rings to matrices. 
\subsection{Categories}
\theoremstyle{definition}\newtheorem*{categorydef}{Category}
We will first take the purely mathematical introduction to category theory. We see that most structures in mathematics have similar key features: a collection of elements (typically with some rules governing them depending on the definition of the structure) and morphisms or transformations between them preserving the structure of elements. Notice, groups and group homomorphisms, rings and ring homomorphisms, topological spaces and continuous maps. This will be our inspiration while defining categories, ultimately the abstraction of these structures. \todo{introduce the non maths way of looking at it as well from the categories, types and structures book}
\begin{categorydef}
  A \emph{category} \cat{C} is a set\footnote{In more rigorous definitions one must be careful of defining the collections of objects as a set lest Russell's paradox comes into play}\todo{Make sure that this is correct.} of \emph{objects} \objs{C}, such as \obj{a}, \obj{b}, \obj{c}, and \emph{morphisms} (or \emph{arrows}) \morphs{C} between them, such as \morph{f}, \morph{g}. We require that:
  \begin{itemize}
    \item There are two operations; \emph{domain} which associates with every arrow \morph{f} an object $\obj{a} = \domain{f}$ and \emph{codomain} which associates with every arrow \morph{f} an object $\obj{b} = \codomain{f}$. We can now express this information as $\morph{f}: \obj{a} \to \obj{b}$.\footnote{Though we emphasise the distinction between a function and a morphism.}
    \item There is a composition rule between morphisms such that given $\morph{f}: \obj{a} \to \obj{b}$ and $\morph{g}: \obj{b} \to \obj{c}$, there is another arrow $\morph{g} \circ \morph{f}: \obj{a} \to \obj{c}$ in \morphs{C}.
    \item Composition of arrows is associative. That is, for an additional object \obj{d} and arrow $\morph{h}: \obj{c} \to \obj{d}$ the resulting morphisms $\morph{h} \circ \left(\morph{g} \circ \morph{f}\right)$ and $\left(\morph{h} \circ \morph{g}\right) \circ \morph{f}$ coincide in \morphs{C}.
    \item Every object \obj{a} is assigned an arrow $\id{a}: \obj{a} \to \obj{a}$ in \morphs{C}, called the \emph{identity morphism}.
    \item Composition with the identity morphism is the identity on morphisms. Explicitly, given the arrow $\morph{f}: \obj{a} \to \obj{b}$, we have $\morph{f} \circ \id{a} = \id{b} \circ \morph{f} = \morph{f}$.
  \end{itemize}
\end{categorydef}
\todo{introduce hom-set}

\subsection{Functors}
\theoremstyle{definition}\newtheorem*{covfunctordef}{Functor}

Similar to the arrows we have within categories, we can transform categoies to other categories. This is the notion of a functor.
\begin{covfunctordef}
  Given two categories \cat{C} and \cat{D}, a (covariant) functor $\functor{F}: \cat{C} \to \cat{D}$ is two related functions, an \emph{object function} and an \emph{arrow function} (both written as \functor{F}). The object function $\functor{F}: \objs{C} \to \objs{D}$ maps an object $a$ in \cat{C} to $\functorobj{F}{a}$ in \cat{D}. Given an arrow $\morph{f}: \obj{a} \to \obj{b}$ in \cat{C}, the arrow function maps \morph{f} to $\functormorph{F}{f}: \functorobj{F}{a} \to \functorobj{F}{b}$. The functor must obey the following properties: \todo{Should I say the arrow is in the cat or the morphs of the cat}
  \begin{itemize}
    \item It must preserve the identity morphism, i.e. $\functormorph{F}{\id{a}} = \id{\functorobj{F}{a}}$.
    \item It must also preserve composition, i.e. $\functormorph{F}{\left(\morph{g} \circ \morph{f}\right)} = \functormorph{F}{\morph{g}} \circ \functormorph{F}{\morph{f}}$.
  \end{itemize}
\end{covfunctordef}

Intuitively, it should be obvious that categories and functors themselves form a category, \commoncatname{Cat}.\\

\subsection{Natural transformations}
\subsection{Adjunctions}
\todo{Describe why adjunctions are such a key character in this paper}
\todo{Add more information between here}
\subsection{Graded Monads}\label{sec:gradedmonads}