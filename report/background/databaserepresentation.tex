\section{Evolution of database representation}
\subsection{Bags}
\paragraph{Characteristics of a database}We expect our database approximation to not be ordered and admit multiplicities and a finite bag of values is one of the simplest constructions that does so. Like a finite set, a bag contains a collection of unordered values. However, unlike a set, bags can contain duplicate elements \cite{RelationalAlgebraByWayOfAdjunctions}.  This multiplicity is key for processing non-idempotent aggregations. For instance, if summing up the ages of a database of people, without admitting multiplicity we would only sum each unique age once.
\subparagraph{Generalisation}Furthermore, going forward we generalise to bags of any types instead of the classical ``bags of records''. This also allows us to deal with intermediate tables that contain non-record values.

In \fref{tab:BagRelAlgOps} we summarise the implementation of relational algebra operators with bags
as their bulk type \cite{RelationalAlgebraByWayOfAdjunctions}.
\begin{table}[h]
    \centering
    \begin{tabular}{r|l}
        table of $V$ values & \bag{V} \\
        empty table & \emptybag \\
        singleton table & \singletonbag \\
        union of tables & $\bagunion{}{}$ \\
        Cartesian product of tables & $\times$ \\
        neutral element & $\lbag () \rbag$ \\
        projection $\projsymb{f}$ & $\bag{f}$ \\
        selection $\selectsymb{p}$ & $filter\ p$ \\
        aggregation in monoid $\monoid{M}$ & $reduce\ \monoid{M}$\\
    \end{tabular}
    \caption{Relational algebra operators implemented for bags}
    \label{tab:BagRelAlgOps}
\end{table}

\subsection{Indexed tables}
We want to move towards an indexed representation of our table in order to equijoin by indexing. \todo{Understand if this is right and equijoin by indexing}. So in this section we introduce the mathematical concepts required to define such an implementation.
\theoremstyle{definition}\newtheorem*{psetdef}{Pointed set}
\theoremstyle{definition}\newtheorem*{ppfuncdef}{Point-preserving function}
\theoremstyle{definition}\newtheorem*{mapdef}{Map}
\theoremstyle{definition}\newtheorem*{finitemapdef}{Finite map}
\begin{psetdef}\label{def:pset}
  A pointed set $\pset{A}{a}$ is a set $A$ with a distinguished element $a \in A$.
\end{psetdef}
We commonly refer to the distinguished element of a set $A$ as $null_A$ or, when not ambiguous to do so, $null$.
\begin{ppfuncdef}\label{def:ppfunc}
  Given two pointed sets $\pset{A}{null_A}$ and $\pset{B}{null_B}$, a total function $f: A \rightarrow B$ is point-preserving if $f(null_A) = null_B$.
\end{ppfuncdef}
\todo{See if there is a way to have better spacing in brackets}

\todo{Add in more mathematical detail for point preserving functions}

We now have the mathematical tools required to define a map. In its finite form a map is widely known in computer science by many other names such as a dictionary, association lists or key-value maps.

Let $\keyset$ be a set and $\valset$ a pointed set. To those already familiar with maps, it may help to think of $\keyset$ as keys and $\valset$ as values.
\begin{mapdef}
  A map of type $\map{\keyset}{\valset}$ is a total function from K to V.
\end{mapdef}
\begin{finitemapdef}
  A finite map of type \finitemap{\keyset}{\valset} is a map where only a finite number of keys are mapped to $null_\valset$ (where $null_\valset$ is the distinguished element of \valset). 
\end{finitemapdef}
The advantage of using a finite map in a database is to allow aggregation.
\todo{Understand why only semi-monoidal}
\todo{Introduced indexed tables}
\paragraph{Useful functions}{} \todo{Explain all the functions needed, such as
merge\label{sec:finitemapfuncs}}

