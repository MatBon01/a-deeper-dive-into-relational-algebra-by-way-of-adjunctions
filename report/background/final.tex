\chapter{Background}
\begin{comment}
The background section of the report should set the project into context by relating it to existing published work which you read at the start of the project when your approach and methods were being considered. There are usually many ways of solving a given problem, and you shouldn't just pick one at random. Describe and evaluate as many alternative approaches as possible. The published work may be in the form of research papers found in the academic literature, articles, text books, technical manuals, or even existing software or hardware of which you have had hands-on experience. Your must acknowledge the sources of your inspiration. You are expected to have seen and thought about other people's ideas; your contribution will be putting them into practice in some other context. However, avoid plagiarism: if you take another person's work as your own and do not cite your sources of information/inspiration you are being dishonest. When referring to other pieces of work, cite the sources where they are referred to or used, rather than just listing them at the end. Accidental plagiarism or not knowing how to cite and reference is not a valid reason for plagiarism. Make sure you read and digest the Department's plagiarism document .

In writing the Background chapter you must demonstrate your ability to analyse, synthesise and apply critical judgement. Analysis is shown by explaining how the proposed solution operates in your own words as well as its benefits and consequences. Synthesis is shown through the organisation of your Related Work section and through identifying and generalising common aspects across different solutions. Critical judgement is shown by discussing the limitations of the solutions proposed both in terms of their disadvantages and limits of applicability.

Typically you can look for Background work using different search engines including:

Google Scholar
IEEExplore
ACM Digital Library
Citeseer
Science Direct
Note 1: Often the terms Background, Literature Review, Related Work and State of the Art are used interchangeably.
Note 2: Keyword search is wonderful, but you need the right Keywords.
Note 2: IEEExplore, ACM Digital Library and Science Direct may require you to be on the College network to download the PDF versions of papers. If at home, use VPN.
\end{comment}

% Currently add in only to stop reference error
\subsection{Evolution of database representation}
\subsubsection{Bags}
\todo{Understand and distinguish between bags being the bulk type}
\paragraph{Characteristics of a database}We expect our database approximation to not be ordered and admit multiplicities and a finite bag of values is one of the simplest constructions that does so. Like a finite set, a bag contains a collection of unordered values. However, unlike a set, bags can contain duplicate elements. \cite{RelationalAlgebraByWayOfAdjunctions} This multiplicity is key for processing non-idempotent aggregations. For instance, if summing up the ages of a database of people, without admitting multiplicity we would only sum each unique age once.
\subparagraph{Generalisation}Furthermore, going forward we generalise to bags of any types instead of the classical ``bags of records''. This also allows us to deal with intermediate tables that contain non-record values.
\todo{Add the mathematical parts about bags}
\todo{When reading about finite maps it says that it's better for databases as non finite maps cannot be aggregated, can non finite bags be aggregates - is this why they are finite?}

\subsubsection{Indexed tables}
We want to move towards an indexed representation of our table in order to equijoin by indexing. \todo{Understand if this is right and equijoin by indexing}. So in this section we introduce the mathematical concepts required to define such an implementation.
\theoremstyle{definition}\newtheorem*{psetdef}{Pointed set}
\theoremstyle{definition}\newtheorem*{ppfuncdef}{Point-preserving function}
\theoremstyle{definition}\newtheorem*{mapdef}{Map}
\theoremstyle{definition}\newtheorem*{finitemapdef}{Finite map}
\begin{psetdef}\label{def:pset}
  A pointed set $\pset{A}{a}$ is a set $A$ with a distinguished element $a \in A$.
\end{psetdef}
We commonly refer to the distinguished element of a set $A$ as $null_A$ or, when not ambiguous to do so, $null$.
\begin{ppfuncdef}\label{def:ppfunc}
  Given two pointed sets $\pset{A}{null_A}$ and $\pset{B}{null_B}$, a total function $f: A \rightarrow B$ is point-preserving if $f(null_A) = null_B$.
\end{ppfuncdef}
\todo{See if there is a way to have better spacing in brackets}

\todo{Add in more mathematical detail for point preserving functions}

We now have the mathematical tools required to define a map. In its finite form a map is widely known in computer science by many other names such as a dictionary, association lists or key-value maps.

Let $\keyset$ be a set and $\valset$ a pointed set. To those already familiar with maps, it may help to think of $\keyset$ as keys and $\valset$ as values.
\begin{mapdef}
  A map of type $\map{\keyset}{\valset}$ is a total function from K to V.
\end{mapdef}
\begin{finitemapdef}
  A finite map of type \finitemap{\keyset}{\valset} is a map where only a finite number of keys are mapped to $null_\valset$ (where $null_\valset$ is the distinguished element of \valset). 
\end{finitemapdef}
The advantage of using a finite map in a database is to allow aggregation.
\todo{Understand why only semi-monoidal}
\todo{Introduced indexed tables} 
