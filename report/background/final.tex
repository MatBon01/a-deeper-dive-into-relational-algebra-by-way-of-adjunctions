\chapter{Background}
\begin{comment}
The background section of the report should set the project into context by relating it to existing published work which you read at the start of the project when your approach and methods were being considered. There are usually many ways of solving a given problem, and you shouldn't just pick one at random. Describe and evaluate as many alternative approaches as possible. The published work may be in the form of research papers found in the academic literature, articles, text books, technical manuals, or even existing software or hardware of which you have had hands-on experience. Your must acknowledge the sources of your inspiration. You are expected to have seen and thought about other people's ideas; your contribution will be putting them into practice in some other context. However, avoid plagiarism: if you take another person's work as your own and do not cite your sources of information/inspiration you are being dishonest. When referring to other pieces of work, cite the sources where they are referred to or used, rather than just listing them at the end. Accidental plagiarism or not knowing how to cite and reference is not a valid reason for plagiarism. Make sure you read and digest the Department's plagiarism document .

In writing the Background chapter you must demonstrate your ability to analyse, synthesise and apply critical judgement. Analysis is shown by explaining how the proposed solution operates in your own words as well as its benefits and consequences. Synthesis is shown through the organisation of your Related Work section and through identifying and generalising common aspects across different solutions. Critical judgement is shown by discussing the limitations of the solutions proposed both in terms of their disadvantages and limits of applicability.

Typically you can look for Background work using different search engines including:

Google Scholar
IEEExplore
ACM Digital Library
Citeseer
Science Direct
Note 1: Often the terms Background, Literature Review, Related Work and State of the Art are used interchangeably.
Note 2: Keyword search is wonderful, but you need the right Keywords.
Note 2: IEEExplore, ACM Digital Library and Science Direct may require you to be on the College network to download the PDF versions of papers. If at home, use VPN.
\end{comment}

\section{The relational model of a database}
We briefly describe the relational model of a database so that we can introduce the key operators we are modelling using category theory. \todo{not modelling, think of better word}
\subsection{Introduction to the relational model}
There are several different data models to choose from when designing a database that specify important aspects of the design such as the structure, operations and constraints on the data \cite{DatabaseSystems}. For this project we concern ourselves with the relational model and its associated algebra.

The relational model represents data through two dimensional tables, called \emph{relations}. Each relation contains several \emph{attributes}, denoting the columns of the table. We call the name of the relation and the set of its attributes a \emph{schema} and they are denoted by the name of the relation followed by the set of attributes in parentheses.\cite{DatabaseSystems} For instance: 
\begin{center}
\begin{verbatim}
  R(a1, a2,..., an)
\end{verbatim}
\end{center}
is a relation \verb|R| with $n$ attributes \verb|a1|, \verb|a2|, \ldots, \verb|an|. A more practical example is the schema
\begin{figure}[!h]
\begin{verbatim}
  People(firstName, surname, age)
\end{verbatim}
\caption[Schema for the People relation]{Example of a schema for the relation People}
\label{fig:peopleSchema}
\end{figure}
This describes a schema for the relation \verb|People| and the attributes \verb|firstName|, \verb|surname| and \verb|age|.
An empty table showing the relation can be found on \fref{tab:peopleRelationHeadings}.
\begin{table}[h]
  \centering
  \begin{tabular}{l|l|l}
    \verb|firstName| & \verb|surname| & \verb|age| \\
    \hline\hline
    & &\\
  \end{tabular}
  \caption{A tabular representation of the People relation.}
  \label{tab:peopleRelationHeadings}
\end{table}
\todo{Add verb to People in the caption}

\subsection{Relational Algebra}
\subsubsection{Projections}
\subsubsection{Selections}
\subsubsection{Products}
\subsubsection{Joins}
\section{Benchmarking databases}
Databases are pervasive to modern society and thus standards have arisen over
the last few decades to ensure that customers are able to pick their preferred
DBMS vendor.

At the top level, database benchmarks are classified into three categories: industry-standard, vendor and
customer-application \cite{PractitionersIntroduction}. 
These classifications are usually motivated by intention of the benchmark
instead of structure of the database management system; there is no shortage of
papers emphasising the importance of domain-specific benchmarks for applications
\cite{PractitionersIntroduction, BenchmarkHandbook} and depending on the
risk/performance tolerance of the application it may be necessary to consider
results from all three categories.

\paragraph{Vendor benchmark} A vendor database benchmark is used by
the database vendor during the production of the database management system.
This benchmark usually serves multiple purposes not just limited to the design
of the system. Of course, it is often used to highlight any performance
bottlenecks driving the design of the system internally but it usually doubles
up and acts as the knowledge base for the marketing of the system. Vendor
benchmarks are usually characterised by a more comprehensive suite of
tests to generate the insights capable of driving the direction of the
product \cite{PractitionersIntroduction}.

\paragraph{Industry-standard benchmark} An industry-standard benchmark is a set
benchmarking suite designed independently to any vendor or solution. It is
designed to allow for a fair comparison between different vendors and has been
shown to increase competition between vendors \cite{Wisconsin2}. As much of
this review will show, many industry-standard benchmarks have been developed
to give results for a wide range of applications of databases for much more
relevant and specialised results; and, although many benchmarks have similar
metrics \cite{SetQueryBenchmark, DebitCredit}, just selecting what metrics to
show consumers is not a trivial task \cite{DebitCredit}. When Gray's paper
\cite{BenchmarkHandbook} was written, it was noted that these benchmarks were
becoming so popular that vendors were also beginning to report their results
with their marketing.

\paragraph{Consumer-application benchmark} This type of benchmark refers to any
benchmarking that a customer would run, typically to choose between different
vendors for their application. This kind of testing can be critical for a
performance-sensitive application \cite{PractitionersIntroduction} and is often
done to test the performance of the database under a specific installation
profile \cite{DoingYourOwnBenchmark} or loads. This is a very specialised
requirements based benchmarking.

The review now turns to different domain-specific benchmarking of databases in
order to comment on and analyse any common structures found when designing
databases. This will help inform major design decisions of this project when
deciding what aspects to include in the benchmarking of the alternative join
syntax.

Despite the innumerable use cases for databases in the modern world, many
applications require and prioritise similar values when choosing a solution.
Furthermore, as most industrial-standard database benchmarks have been designed in
order to allow comparison for specific domain requirements, you will commonly
find database benchmarks that are designed with one of two different
applications in mind: \emph{transaction processing} or \emph{decision support}
\cite{PractitionersIntroduction}; it is worth mentioning that other application
types exist, however, such as \emph{document search} and \emph{direct marketing}
\cite{SetQueryBenchmark}. The application types of transaction processing and
decision support are so pervasive that they are commonly also used to partition
other application types, for instance OLTP (Online transaction processing) and
OLAP (Online analytical processing) are specialised for online processing but
share many comparative similarities as transaction processing and decision
support respectively \cite{OLTP-Oracle}.

This review will provide a brief overview of both
the main database application types, then focus more strongly on decision
support benchmarks as ad-hoc query processing is more heavily tested in these
benchmarks and thus more relevant to the project.

\subsection{Transaction processing}
Transaction processing is characterised by a large number of
update-intensive database services with a particular emphasis on integrity of
its requests \cite{PractitionersIntroduction}.
It is clear that this type of environment
demands an emphasis on throughput and integrity. For instance, a bank would
likely have an online transaction processing system in place and the
ability to deal with a large number of transactions in a short period of
time while maintaining the accuracy of the information is most likely paramount.
There are many well known benchmarks to test these types of systems,
including \emph{DebitCredit}, \emph{TPC-C} and \emph{TPC-E} \cite{DebitCredit,
TPC-OLTP}.
These benchmarks usually measure transactions per second or price per
transactions per second. Efforts have been made to try and describe what is
meant by ``price'' and usually it refers to the five-year capital cost (e.g.
cost of hardware, software and system maintenance over a five year period)
\cite{BenchmarkHandbook, DebitCredit}. The advantage of giving a metric
dependent on price is that price is likely to be a limiting of important factor
to most (if not all) organisations when shopping for database solutions and so
it helps encompass all information about the system into a single comparable
number \cite{SetQueryBenchmark}. It is worth noting that many benchmarks which
include this metric also give more information for the user \cite{TPC-H} or
allow the user to weight the comparative importance of features in this metric
\cite{SetQueryBenchmark} but it is expected that much emphasis will come down to
this defining number \cite{SetQueryBenchmark}.

We note that the above definition of a transaction processing system not only
highlights its intense performance requirements but also notes the importance of
integrity. The bank example highlights intuitively what we mean by integrity, a
shop may not want the transaction to go through zero times if it is reported
that it has, in fact, been processed, just as I would not want the transaction
to go through twice. So what exactly is a transaction? And how can we specify
its integrity in a rigorous way?

A transaction simply is an input to the database that must be considered as one
unit of work \cite{ComputerScienceDictionary} and completed independently to any
concurrent actions occurring \cite{ACID}. The integrity of this behaviour can be described 
by a set of properties typically remembered as the acronym \emph{ACID} and
explained below \cite{ACID, PractitionersIntroduction}.

\paragraph{Atomicity} An ``all or nothing'' property of a transaction. All
individual operations must take place or any partially-completed sequence of
operations must have no lasting effects on the data. The user should be informed
of which outcome has occurred.

\paragraph{Consistency} Given the database was in a consistent/valid state
before the transaction processed, it must remain in a consistent/valid state
after any changes are committed; that is to say the changes cause not integrity
constraints to be violated \cite{IntroToDatabaseSystems}. 
Some texts favour the term ``correctness'' \cite{IntroToDatabaseSystems} but this
discussion is outside of the scope of the project. 
% correctness vs consistency: correctness may be preferred as it is stronger, 
% and says that the database only contains true statements about the world

\paragraph{Isolation} Information related to the execution of one transaction
should be hidden from other transactions running concurrently. This is
equivalent to being able to find an order such that the results of the execution
of all transactions would be the same if they were executed sequentially in that
order.

\paragraph{Durability} Once a transaction has been successfully completed, its
effects must persist and future malfunctions.

\subsection{Decision support}
Decision support is characterised by the need to execute complicated queries on
a database with fewer changes \cite{IntroToDatabaseSystems,
PractitionersIntroduction}. The business value from this sort of application are
vast, especially in automatic report generation to produce insights and direct
company strategy \cite{SetQueryBenchmark, IntroToDatabaseSystems}. Despite the
clear business needs, the measurements usually taken by decision support
benchmarks are not as standard as those by transaction processing benchmarks and
metrics usually vary according to benchmark 
\cite{PractitionersIntroduction}; some common metrics collected are a measure of
elapsed time \cite{Wisconsin, TPC-H, PractitionersIntroduction}, performance of
the underlying system (CPU and I/O utilisation) \cite{PractitionersIntroduction}
and the throughput or price per throughput \cite{TPC-H, SetQueryBenchmark,
PractitionersIntroduction} (as in transaction support systems). 

Although less pervaisive in our every day life to transaction support systems,
there are a large number of benchmarks for decision support systems. To name a
few there are the \emph{Wisconsin benchmark} \cite{Wisconsin}, \emph{Bull Single-User Decision
Support benchmark (SUDS)} \cite{PractitionersIntroduction}, \emph{Set Query
benchmark} \cite{SetQueryBenchmark} and \emph{TPC-H} \cite{TPC-H}. 

\subsection{Database benchmarking case studies}
% Explain why selected these benchmarks

% Wisconsin
% Set query
% DebitCredit

\subsection{Best practices benchmarking}

\section{Category theory}
Category theory will be the main tool in describing the structure and operations
of our database system. Category theory is a very powerful tool in mathematics
and can be seen as an ``abstraction of abstractions''. The category theory that
will appear in this paper is very limited and does not do justice to the full
language, although some mathematical examples are presented to try and aid
visualisation of concepts.

\subsection{Categories}
\theoremstyle{definition}\newtheorem*{categorydef}{Category}
We see that most structures in mathematics have similar key features: a
collection of elements (typically with some rules governing them depending on
the definition of the structure) and morphisms or transformations between them
preserving the structure of elements. Notice, groups and group homomorphisms,
rings and ring homomorphisms, topological spaces and continuous maps. This will
be our inspiration while defining categories, ultimately the abstraction of
these structures. 
\begin{categorydef}
  A \emph{category} \cat{C} is a set\footnote{In more rigorous definitions one must be careful of defining the collections of objects as a set lest Russell's paradox comes into play} of \emph{objects} \objs{C}, such as \obj{a}, \obj{b}, \obj{c}, and \emph{morphisms} (or \emph{arrows}) \morphs{C} between them, such as \morph{f}, \morph{g}. We require that:
  \begin{itemize}
    \item There are two operations; \emph{domain} which associates with every arrow \morph{f} an object $\obj{a} = \domain{f}$ and \emph{codomain} which associates with every arrow \morph{f} an object $\obj{b} = \codomain{f}$. We can now express this information as \explicitmorph{f}{a}{b}.\footnote{Though we emphasise the distinction between a function and a morphism.}
    \item There is a composition rule between morphisms such that given \explicitmorph{f}{a}{b} and \explicitmorph{g}{b}{c}, there is another arrow \explicitmorph{\morph{g} \circ \morph{f}}{a}{c} in \morphs{C}.
    \item Composition of arrows is associative. That is, for an additional object \obj{d} and arrow \explicitmorph{h}{c}{d} the resulting morphisms $\morph{h} \circ \left(\morph{g} \circ \morph{f}\right)$ and $\left(\morph{h} \circ \morph{g}\right) \circ \morph{f}$ coincide in \morphs{C}.
    \item Every object \obj{a} is assigned an arrow $\id{a}: \obj{a} \to \obj{a}$ in \morphs{C}, called the \emph{identity morphism}.
    \item Composition with the identity morphism is the identity on morphisms. Explicitly, given the arrow \explicitmorph{f}{a}{b}, we have $\morph{f} \circ \id{a} = \id{b} \circ \morph{f} = \morph{f}$.
  \end{itemize}
\end{categorydef}

\paragraph{}Furthermore, a \emph{hom-set} \homset{C}{\obj{a}}{\obj{b}} is
defined as the collection of all arrows in \cat{C} with domain \obj{a} and
codomain \obj{b}.

\subsection{Functors}
\theoremstyle{definition}\newtheorem*{covfunctordef}{Functor}

Similar to the arrows we have within categories, we can transform categories to other categories. This is the notion of a functor.
\begin{covfunctordef}
  Given two categories \cat{C} and \cat{D}, a (covariant) functor $\functor{F}:
  \cat{C} \to \cat{D}$ is two related functions, an \emph{object function} and
  an \emph{arrow function} (both written as \functor{F}). The object function
  $\functor{F}: \objs{C} \to \objs{D}$ maps an object $a$ in \cat{C} to
  $\functorobj{F}{a}$ in \cat{D}. Given an arrow $\morph{f}: \obj{a} \to
  \obj{b}$ in \cat{C}, the arrow function maps \morph{f} to
  $\functormorph{F}{f}: \functorobj{F}{a} \to \functorobj{F}{b}$ in \cat{D}. The functor must obey the following properties:
  \begin{itemize}
    \item It must preserve the identity morphism, i.e. $\functormorph{F}{\id{a}} = \id{\functorobj{F}{a}}$.
    \item It must also preserve composition, i.e.
        $\functormorph{F}{\left(\morph{g} \circ \morph{f}\right)} = \functormorph{F}{\morph{g}} \circ \functormorph{F}{\morph{f}}$.
  \end{itemize}
\end{covfunctordef}

Intuitively, it should be obvious that categories and functors themselves form a category, \commoncatname{Cat}.\\

\subsection{Natural transformations}
\theoremstyle{definition}\newtheorem*{nattransdef}{Natural Transformation}
We now can explore an intuitive way of relating two functors, called an \emph{natural transformation}. A natural transformation can be seen as a projection of one functor space to another and is in essence a family of arrows that describes this translation.
\begin{nattransdef}
Given two functors $\functor{F}, \functor{G} : \cat{C} \to \cat{D}$ a
\emph{natural transformation} $\explicitnattrans{\tau}{F}{G}$ associates to each element $\obj{a} \in \objs{C}$ an arrow $\nattransapply{\tau}{a} \in \morphs{D}$ such that $\nattransapply{\tau}{a}: \functorobj{F}{a} \to \functorobj{G}{a}$. Additionally, for $\obj{b} \in \cat{C}$ and \explicitmorph{f}{a}{b} a morphism in \cat{C}, we also require that $\nattransapply{\tau}{b} \circ \functormorph{F}{f} = \functormorph{G}{f} \circ \nattransapply{\tau}{a}$.

\end{nattransdef}

\theoremstyle{definition}\newtheorem*{verticalcompositiondef}{Vertical Composition}

\begin{verticalcompositiondef}
    Given two natural transformations
    \explicitnattrans{\tau}{E}{F} and
    \explicitnattrans{\nu}{F}{G} where $\functor{E, F, G}: \cat{C} \to
    \cat{D}$ are functors. The vertical composition \cite{RelationalAlgebraByWayOfAdjunctions}
    $\explicitnattrans{\verticalcomposition{\tau}{\nu}}{E}{G}$
    is defined by the composition of the underlying arrows for every
    object $\obj{A} \in \cat{C}$. Explicitly, the components
    $\nattransapply{\left(\verticalcomposition{\tau}{\nu}\right)}{A} =
    \morphcomp{\nattransapply{\tau}{A}}{\nattransapply{\nu}{A}}$
\label{sec:verticalcomposition}
\end{verticalcompositiondef}


\subsection{Adjunctions}
An adjunction expresses an intersection of arrows of two different functions. For instance Say you have the category \commoncatname{Set} of sets \cite{RelationalAlgebraByWayOfAdjunctions} and two functions $\morph{f}, \morph{g}$ with signatures \explicitmorph{f}{X}{A} and \explicitmorph{g}{X}{B} with $\obj{X}, \obj{A}, \obj{B} \in \commoncatname{Set}$.
in \commoncatname{Set} with a common domain $\domain{f} = \domain{g} = \obj{A}
\in \commoncatname{Set}$, how might we interpret the application of both
functions on one element. We could create a new function \explicitmorph{h}{X}{A
\times B} which for all $a \in \obj{A}$ maps $a \mapsto (\morph{f}(a),
\morph{g}(a))$. Note that both \obj{X} and $\obj{A} \times \obj{B}$ are sets
(where $\times$ denotes the Cartesian product of two sets, i.e. the set of all
such pairings of elements of \obj{A} and \obj{B}). This is not a very natural
way of writing this however. Consider now the product category
$\commoncatname{Set}^2$ such that the elements are just pairs of sets and
functions, pairs of functions $(\morph{i}, \morph{j}): (\domain{i}, \domain{j})
\to (\codomain{i}, \codomain{j})$. It is clear that \commoncatname{Set} and
$\commoncatname{Set}^2$ are distinct, i.e. a pair of a set is not a set nor is a
pair of set. However, interestingly $h$ can quite easily be expressed within
$\commoncatname{Set}^2$ as the function $(f, g)$ that maps $(\obj{A}, \obj{A})$
to $(\obj{X}, \obj{Y})$ and links back to the situation in $\commoncatname{Set}$
become clear. We introduce the diagonal functor $\functor{\Delta}:
\commoncatname{Set} \to \commoncatname{Set}^2$, s.t. $\functorobj{\Delta}{A} =
(\obj{A}, \obj{A})$ and $\functormorph{\Delta}{f} = (\morph{f}, \morph{f})$.
Furthermore, the Cartesian product can also be seen view the lens of a functor $\functor{\times}: \commoncatname{Set}^2 \to \commoncatname{Set}$ that takes the pair of elements $(\obj{A}, \obj{B})$ and maps it to the set $\obj{A} \times \obj{B}$ and a pair of functions $(\morph{i}, \morph{j})$ to a function $k: \left(\domain{i} \times \domain{j}\right) \to \left(\codomain{i} \times \codomain{j}\right)$.
Considering the above problem again, we can see very clear links between the two
functors. We can work our problem in \commoncatname{Set} in the domain of
$\commoncatname{Set}^2$ by considering $\functorobj{\Delta}{A} = (\obj{A},
\obj{A})$. We can then easily apply $(\morph{f}, \morph{g})$. Similarly, given
the problem in $\commoncatname{Set}^2$ we see that we are left with a pair of
sets that can easily be considered a Cartesian product $\obj{X} \times \obj{Y}$. It should be clear that there is a bijection between the arrows $\functorobj{\Delta}{A} \to (\obj{X}, \obj{Y})$ in $\commoncatname{Set}^2$ and the arrows $\obj{A} \to \functorobj{(\times)}{(X, Y)}$.

\theoremstyle{definition}\newtheorem*{adjunctiondef}{Adjuction}
Motivated by this example, we give a formal definition of an adjunction.
\begin{adjunctiondef}
    Given two functors $\functor{L}: \cat{D} \to \cat{C}$ and $\functor{R}:
    \cat{C} \to \cat{D}$, we define an adjunction $\adunction{L}{R}$ \todo{has a
    domain and codomain} such that there is a natural isomorphism between the
    hom-sets as follows:\cite{RelationalAlgebraByWayOfAdjunctions}
    \[
        \lfloor - \rfloor: \homset{C}{\functorobj{L}{A}}{B} \cong
        \homset{D}{A}{\functorobj{R}{B}} :\lceil - \rceil
    \]
    We call \functor{L} the \emph{left adjoint} and \functor{R} the \emph{right
    adjoint}.
    The natural transformations between the arrows of the categories are defined
    by the \emph{unit}: \explicitnattrans{\eta}{\mathrm{Id}}{R \circ L}, $\nattransapply{\eta}{A} =
    \lfloor \id{\functorobj{L}{A}} \rfloor$ and, symmetrically, the
    \emph{counit}:
    $\explicitnattrans{\epsilon}{L \circ R}{\mathrm{Id}}$ such that
    $\nattransapply{\epsilon}{B} = \lceil \id{\functorobj{R}{A}} \rceil$. We
    require that the unit and counit obey the `triangle identities':
\[
            \verticalcomposition{\nattransapply{\eta}{\functor{R}}}{\functorobj{R}{\nattrans{\epsilon}}}
            = \id{}
\] \todo{fix notation}
    and
\[
    \verticalcomposition{\functorobj{L}{\nattrans{\eta}}}{\nattransapply{\epsilon}{\functor{L}}}
    = \id{}
\]
\end{adjunctiondef}

We see that the unit $\eta$ is a mapping between elements in \cat{D}, noting
especially the difference between the functor \functor{\mathrm{Id}} and the
morphism \id{\functorobj{L}{A}}.
In trying to understand the triangle inequalities it is worth remembering the
definition of vertical composition in \fref{sec:verticalcomposition}, namely
that
$\nattransapply{\left(\verticalcomposition{\eta}{\epsilon}\right)}{\obj{A}} =
\morphcomp{\nattransapply{\eta}{\obj{A}}}{\nattransapply{\epsilon}{\obj{A}}}$
where the right hand side is simple the composition of arrows.

\subsection{Evolution of database representation}
\subsubsection{Bags}
\todo{Understand and distinguish between bags being the bulk type}
\paragraph{Characteristics of a database}We expect our database approximation to not be ordered and admit multiplicities and a finite bag of values is one of the simplest constructions that does so. Like a finite set, a bag contains a collection of unordered values. However, unlike a set, bags can contain duplicate elements. \cite{RelationalAlgebraByWayOfAdjunctions} This multiplicity is key for processing non-idempotent aggregations. For instance, if summing up the ages of a database of people, without admitting multiplicity we would only sum each unique age once.
\subparagraph{Generalisation}Furthermore, going forward we generalise to bags of any types instead of the classical ``bags of records''. This also allows us to deal with intermediate tables that contain non-record values.
\todo{Add the mathematical parts about bags}
\todo{When reading about finite maps it says that it's better for databases as non finite maps cannot be aggregated, can non finite bags be aggregates - is this why they are finite?}

\subsubsection{Indexed tables}
We want to move towards an indexed representation of our table in order to equijoin by indexing. \todo{Understand if this is right and equijoin by indexing}. So in this section we introduce the mathematical concepts required to define such an implementation.
\theoremstyle{definition}\newtheorem*{psetdef}{Pointed set}
\theoremstyle{definition}\newtheorem*{ppfuncdef}{Point-preserving function}
\theoremstyle{definition}\newtheorem*{mapdef}{Map}
\theoremstyle{definition}\newtheorem*{finitemapdef}{Finite map}
\begin{psetdef}\label{def:pset}
  A pointed set $\pset{A}{a}$ is a set $A$ with a distinguished element $a \in A$.
\end{psetdef}
We commonly refer to the distinguished element of a set $A$ as $null_A$ or, when not ambiguous to do so, $null$.
\begin{ppfuncdef}\label{def:ppfunc}
  Given two pointed sets $\pset{A}{null_A}$ and $\pset{B}{null_B}$, a total function $f: A \rightarrow B$ is point-preserving if $f(null_A) = null_B$.
\end{ppfuncdef}
\todo{See if there is a way to have better spacing in brackets}

\todo{Add in more mathematical detail for point preserving functions}

We now have the mathematical tools required to define a map. In its finite form a map is widely known in computer science by many other names such as a dictionary, association lists or key-value maps.

Let $\keyset$ be a set and $\valset$ a pointed set. To those already familiar with maps, it may help to think of $\keyset$ as keys and $\valset$ as values.
\begin{mapdef}
  A map of type $\map{\keyset}{\valset}$ is a total function from K to V.
\end{mapdef}
\begin{finitemapdef}
  A finite map of type \finitemap{\keyset}{\valset} is a map where only a finite number of keys are mapped to $null_\valset$ (where $null_\valset$ is the distinguished element of \valset). 
\end{finitemapdef}
The advantage of using a finite map in a database is to allow aggregation.
\todo{Understand why only semi-monoidal}
\todo{Introduced indexed tables} 
