\section{The relational model of a database}
We briefly describe the relational model of a database so that we can introduce the key operators we are modelling using category theory. \todo{not modelling, think of better word}
\subsection{Introduction to the relational model}
There are several different data models to choose from when designing a database that specify important aspects of the design such as the structure, operations and constraints on the data \cite{DatabaseSystems}. For this project we concern ourselves with the relational model and its associated algebra.

The relational model represents data through two dimensional tables, called \emph{relations}. Each relation contains several \emph{attributes}, denoting the columns of the table. We call the name of the relation and the set of its attributes a \emph{schema} and they are denoted by the name of the relation followed by the set of attributes in parentheses.\cite{DatabaseSystems} For instance: 
\begin{center}
\begin{verbatim}
  R(a1, a2,..., an)
\end{verbatim}
\end{center}
is a relation \verb|R| with $n$ attributes \verb|a1|, \verb|a2|, \ldots, \verb|an|. A more practical example is the schema
\begin{figure}[!h]
\begin{verbatim}
  People(firstName, surname, age)
\end{verbatim}
\caption[Schema for the People relation]{Example of a schema for the relation People}
\label{fig:peopleSchema}
\end{figure}
This describes a schema for the relation \verb|People| and the attributes \verb|firstName|, \verb|surname| and \verb|age|.
An empty table showing the relation can be found on \fref{tab:peopleRelationHeadings}.
\begin{table}[h]
  \centering
  \begin{tabular}{l|l|l}
    \verb|firstName| & \verb|surname| & \verb|age| \\
    \hline\hline
    & &\\
  \end{tabular}
  \caption{A tabular representation of the People relation.}
  \label{tab:peopleRelationHeadings}
\end{table}
\todo{Add verb to People in the caption}

A row of the table\footnote{Excluding the row with the headings.} is called a \emph{tuple} and it consists of \emph{components}, one per attribute. A tuple is simply denoted by a comma separated list of its components within parentheses. For instance a tuple for the \verb|People| relation would be:
\begin{figure}[!h]
  \centering
  (\verb|Jim|, \verb|Smith|, 32).
\end{figure}

In the relational model the constraint that each each component of a tuple must be an of an atomic type; meaning the component should reasonably not be a compound data type such as records, lists or sets. Furthermore, each attribute has its own associated \emph{domain}, specifying an elementary type that all components belonging to that column must take. The domain can be specified in the the schema using a colon as follows:
\begin{figure}[!h]
  \begin{verbatim}
    People(firstName: string, surname: string, age: integer).
  \end{verbatim}
\end{figure}

It is natural now to see a relation as a set of tuples, and a database as one or more relations. The \emph{(relational) database schema} is the set of schemas the relations in the database adhere to.\cite{DatabaseSystems}
\todo{Do I need to talk about tuples being built up and a final mention of attributes}

Finally, we call a set of tuples for a relation an \emph{instance} of that relation as by the nature of database systems, it will eventually change. The current set of tuples is called the \emph{current instance}.\cite{DatabaseSystems}

\paragraph{A note on ordering} The distinction between lists and sets here is very important and has consequences. Firstly, note that a schema consists of a \textbf{set} of attributes, though in this case we do assign a ``standard'' ordering to them, typically following the ordering in the schema. Also note that when giving a tuple of a relation we do not also give the headings and thus some indication to which schema it belongs is necessary. Secondly, we introduced a relation as a \textbf{set} of tuples, in this case there is no standard ordering and thus invites many equivalent representations of the same relation.\cite{DatabaseSystems} In generality, the order of columns and rows do not matter, so long as they are consistent.

\subsection{Relational Algebra}
Equipped with the information above we can now introduce the domain of the project, relational algebra. Relational algebra views queries as the creation of a relation by operating on other relations.\cite{RelationalCalculus}

The sections below describe some of the more interesting, specialised operations on this algebra but it is definitely worth noting that as relations are sets \todo{bags?} of tuples, given matching schemas between relations, set operations are also valid operations and produce relations.\cite{RelationalModel} More specifically, you can find the union, intersection and differences of two relations just as you would with sets. \cite{DatabaseSystems}
\subsubsection{Projections}
Projections select certain attributes of a relation, removing the others and collapsing duplicates \todo{I thought we needed multiplicty, check this}.\cite{RelationalModel} This can be visualised as only keeping certain columns in the table.

Given a list of $k$ attributes $L = i_1, i_2, \ldots, i_k$, we can specify a projection on an $n$-ary (containing $n$ attributes) relation \relation{R} as \proj{L}{R}. The result is a $k$-ary relation whose attributes are specified by $L$ \cite{RelationalModel} and by convention the ``standard order'' of attributes is also determined by the order of attributed in $L$.\cite{DatabaseSystems}
\subsubsection{Selections}
\subsubsection{Products}
\subsubsection{Joins}
\subsubsection{Note on permutations}
Permutations is another specialist operation in relational algebra, though not important to the scope of the project. For completion, despite the fact that relations are domain--unordered, their internal representation in computers is not and so permutation may be done for performance benefits despite no logical difference storing a relation and its permutations.\todo{Make sure I worded the performance benefits thing correctly}\cite{RelationalModel} Furthermore, permutation can be used (and is usually implied) to ensure that tuples with identical schemas differing only in ordering can have the normal set operations applied to them. \cite{DatabaseSystems}