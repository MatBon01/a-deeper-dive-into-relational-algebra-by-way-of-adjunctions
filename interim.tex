\documentclass{article}

\usepackage{comment}

\title{A Deeper Dive into Relational Algebra by Way of Adjunctions Interim Report} 
\author{Matteo Bongiovanni}
\date{January 2023}

\begin{document}
\maketitle

\begin{comment}
The aim of the project interim report is multi-fold:

  1. To provide a document that your second marker can use as a basis for discussion on your project plan and progress to date.
  2. To show that you have considered the ethical implications of your project.
  3. To provide a substantial body of text, primarily the project background and related work, that you can use in your final report. 

By the time the interim report is due you should have a clearly defined project, understand well the motivation and issues to be addressed, know the background work in detail, have the main ideas for how to tackle the problem and have started the development. You should also have a plan for the remainder of the project and, importantly, how to evaluate the project.

The interim report should contain the following sections. An approximate page count is suggested for each section, but there are no hard limits either way:


You are free to write up any additional material that will appear in the final report, for example a section or chapter describing a significant component of the design/implementation that you have already completed.  Avoid any additional material that is not re-usable in the final report.

As always, use diagrams and examples (e.g. code) wherever appropriate.

If you need inspiration, take a look at the Distinguished Projects from previous years, focusing in particular at this stage on the introduction, background and evaluation sections.
\end{comment}

\section{Introduction} % 1-3 pages
\begin{comment}
It’s a good idea to *try* to write the introduction to your final report early on in the project. However, you will find it hard, as you won’t yet have a complete story and you won’t know what your main contributions are going to be. However, the exercise is useful as it will tell you what you *don’t* yet know and thus what questions your project should aim to answer. For the interim report this section should be a short, succinct, summary of the project’s main objectives. Some of this material may be re-usable in your final report, but the chances are that your final introduction will be quite different.  You are therefore advised to keep this part of the interim report short, focusing on the following questions: What is the problem, why is it interesting and what’s your main idea for solving it?  (DON'T use those three questions as subheadings however!  The answers should emerge from what you write.)
\end{comment}

\section{Background} % 10-20 pages
\begin{comment}
This should form the bulk of the interim report. You should consider that your objective here is to produce a near final version of the background section, as it will appear in your final report.  All of this material should be re-usable, so it is worth getting it right at this stage of the project.  The details of what to include can be found in the Project Report guidelines.

From the Project Report guidelines:
The background section of the report should set the project into context by relating it to existing published work which you read at the start of the project when your approach and methods were being considered. There are usually many ways of solving a given problem, and you shouldn't just pick one at random. Describe and evaluate as many alternative approaches as possible. The published work may be in the form of research papers found in the academic literature, articles, text books, technical manuals, or even existing software or hardware of which you have had hands-on experience. You must acknowledge the sources of your inspiration. You are expected to have seen and thought about other people's ideas; your contribution will be putting them into practice in some other context. However, avoid plagiarism: if you take another person's work as your own and do not cite your sources of information/inspiration you are being dishonest. When referring to other pieces of work, cite the sources where they are referred to or used, rather than just listing them at the end. Accidental plagiarism or not knowing how to cite and reference is not a valid reason for plagiarism. Make sure you read and digest the Department's plagiarism document .

In writing the Background chapter you must demonstrate your ability to *analyse*, *synthesise* and *apply critical* judgement. Analysis is shown by explaining how the proposed solution operates in your own words as well as its benefits and consequences. Synthesis is shown through the organisation of your Related Work section and through identifying and generalising common aspects across different solutions. Critical judgement is shown by discussing the limitations of the solutions proposed both in terms of their disadvantages and limits of applicability.

Typically you can look for Background work using different search engines including:
* Google Scholar
* IEEExplore
* ACM Digital Library
* Citeseer
* Science Direct

**Note 1:** Often the terms *Background, Literature Review, Related Work* and *State of the Art* are used interchangeably.
**Note 2**: Keyword search is wonderful, but you need the right *Keywords*.
**Note 3:** IEEExplore, ACM Digital Library and Science Direct may require you to be on the College network to download the PDF versions of papers. If at home, use VPN.
\end{comment}

\section{Project Plan} % 1-2 pages
\begin{comment}
You should explain what needs to be done in order to complete the project and roughly what you expect the timetable to be. Don’t forget to include the project write-up (the final report), as this is a major part of the exercise. It’s important to identify key milestones and also fall-back positions, in case you run out of time.  You should also identify what extensions could be added if time permits.  The plan should be complete and should include those parts that you have already addressed (make it clear how far you have progressed at the time of writing).  This material will *not* appear in the final report.
\end{comment}

\section{Evaluation plan} % 1-2 pages
\begin{comment}
Project evaluation is very important, so it's important to think now about how you plan to measure success. For example, what functionality do you need to demonstrate?  What experiments to you need to undertake and what outcome(s) would constitute success?  What benchmarks should you use? How has your project extended the state of the art?  How do you measure qualitative aspects, such as ease of use?  These are the sort of questions that your project evaluation should address; this section should outline your plan.
\end{comment}

\section{Ethical issues} % 1-2 pages
\begin{comment}
What are the wider ethical, legal, professional and societal issues surrounding your project and the accompanying research? You should use the ethics checklist as the basis for this discussion. 
\end{comment}

\section{Bibliography/References} % TODO: check to see if you need a section
\begin{comment}
DoC uses the Vancouver Referencing Format.
\end{comment}


\end{document}